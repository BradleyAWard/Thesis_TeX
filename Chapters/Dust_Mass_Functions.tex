\textit{
Things implicitly assumed in the following that I will write in previous sections:
\begin{enumerate}
    \item The dust cycle
    \item A conclusion on what the full H-ATLAS survey provides
\end{enumerate}
}

\section{Introduction}

\section{Interstellar Dust as a Tracer of Cosmic Evolution}

Interstellar dust is responsible for obscuring approximately half of all starlight since the Big Bang (\citealt{Puget_1996}; \citealt{Fixsen_1998}; \citealt{Dole_2006}; \citealt{Driver_2016}). Assuming that young, massive stars are the main culprit for the heating of the interstellar dust and produce the UV photons that get absorbed and reradiated to far-IR wavelengths (\citealt{Kennicutt_1998}; \citealt{Calzetti_2007}; \citealt{Kennicutt_2009}), then measuring extragalactic dust masses is useful in our understanding of obscured star formation at different times in cosmic history. This naturally results from dust being a by-product of star formation. In order to study the evolution in the star formation rate density of the Universe, the most direct method is to measure the mass of galaxies' molecular gas reservoirs from which the stars form and estimate the star formation activity from the evolution of the cosmic molecular gas mass density. As detailed in the subsequent Chapter, molecular hydrogen, $H_2$ (the main component of the molecular gas), cannot be easily observed from the cold ISM ({\color{red} Typical temperature range}), whereas dust emission provides a suitable alternative at these temperatures. Using independent measures of the cosmic gas mass density from dust-traced gas mass (assuming some gas-to-dust mass ratio) allows us to constrain both the contribution of obscured star formation and the dust content of galaxies over time.
