\textit{
Things implicitly assumed in the following that I will write in previous sections:
\begin{enumerate}
    \item The dust cycle
    \item A conclusion on what the full H-ATLAS survey provides
\end{enumerate}
}

\section{Introduction}

\section{Interstellar Dust as a Tracer of Cosmic Evolution}

Interstellar dust is responsible for obscuring approximately half of all starlight since the Big Bang (\citealt{Puget_1996}; \citealt{Fixsen_1998}; \citealt{Dole_2006}; \citealt{Driver_2016}). Assuming that young, massive stars are the main culprit for the heating of the interstellar dust and produce the UV photons that get absorbed and reradiated to far-IR wavelengths (\citealt{Kennicutt_1998}; \citealt{Calzetti_2007}; \citealt{Kennicutt_2009}), then measuring extragalactic dust masses is useful in our understanding of obscured star formation at different times in cosmic history. This naturally results from dust being a by-product of star formation. In order to study the evolution in the star formation rate density of the Universe, the most direct method is to measure the mass of galaxies' molecular gas reservoirs from which the stars form and estimate the star formation activity from the evolution of the cosmic molecular gas mass density. As detailed in the subsequent Chapter, molecular hydrogen, $H_2$ (the main component of the molecular gas), cannot be easily observed from the cold ISM ({\color{red} Typical temperature range}), whereas dust emission provides a suitable alternative at these temperatures. Using independent measures of the cosmic gas mass density from dust-traced gas mass (assuming some gas-to-dust mass ratio) allows us to constrain both the contribution of obscured star formation and the dust content of galaxies over time.In addition, metals are produced and expelled during supernovae and from stellar winds, with some fraction becoming bound to dust grains in the ISM. Here they may be ejected from the galaxy by starburst winds or become part of the next generation of stars. As a result, the total dust mass is a fundamental property of the ISM that can be used to trace the evolutionary stage of the galaxy, especially when used in conjunction with stellar and gas masses (e.g. \citealt{Cortese_2012}; \citealt{deVis_2017a}; \citealt{deVis_2017b}).

For these reasons, studying the distribution of dust masses for a population of galaxies across time can provide important insights into the properties and evolution of galaxies and their ISM. The dust mass function (DMF), the space density of galaxies as a function of dust mass, is a fundamental measure of the dust content of galaxies, providing crucial information on galaxy evolution, star formation activity and the cosmic dust budget. In this study we use the results of the crossmatching analysis of the South Galactic Pole field of the \textit{Herschel}-ATLAS survey, as presented in the previous Chapter, to derive the far-IR selected DMF and investigate whether it has evolved with redshift. 

\section{Estimating Dust Masses of SGP Galaxies}

As demonstrated previously, counterparts identified from optical wavebands such as the SDSS and GAMA surveys, have optical depths that limit 

[SGP allows us to estimate dust masses of Herschel galaxies to higher redshifts than previously]

[What data we use - any restrictions etc.]

[SED fitting and estimating dust masses]
