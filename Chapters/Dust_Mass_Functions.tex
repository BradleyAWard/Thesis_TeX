\textit{
Things implicitly assumed in the following that I will write in previous sections:
\begin{enumerate}
    \item The dust cycle
    \item A conclusion on what the full H-ATLAS survey provides
\end{enumerate}
}

\section{Introduction}

\section{Interstellar Dust as a Tracer of Cosmic Evolution}

Interstellar dust is responsible for obscuring approximately half of all starlight since the Big Bang (\citealt{Puget_1996}; \citealt{Fixsen_1998}; \citealt{Dole_2006}; \citealt{Driver_2016}). Assuming that young, massive stars are the main culprit for the heating of the interstellar dust and produce the UV photons that get absorbed and reradiated to far-IR wavelengths (\citealt{Kennicutt_1998}; \citealt{Calzetti_2007}; \citealt{Kennicutt_2009}), then measuring extragalactic dust masses is useful in our understanding of obscured star formation at different times in cosmic history. This naturally results from dust being a by-product of star formation. In order to study the evolution in the star formation rate density of the Universe, the most direct method is to measure the mass of galaxies' molecular gas reservoirs from which the stars form and estimate the star formation activity from the evolution of the cosmic molecular gas mass density. As detailed in the subsequent Chapter, molecular hydrogen, $H_2$ (the main component of the molecular gas), cannot be easily observed from the cold ISM ({\color{red} Typical temperature range}), whereas dust emission provides a suitable alternative at these temperatures. Using independent measures of the cosmic gas mass density from dust-traced gas mass (assuming some gas-to-dust mass ratio) allows us to constrain both the contribution of obscured star formation and the dust content of galaxies over time. In addition, metals are produced and expelled during supernovae and from stellar winds, with some fraction becoming bound to dust grains in the ISM. Here they may be ejected from the galaxy by starburst winds or become part of the next generation of stars. As a result, the total dust mass is a fundamental property of the ISM that can be used to trace the evolutionary stage of the galaxy, especially when used in conjunction with stellar and gas masses (e.g. \citealt{Cortese_2012}; \citealt{deVis_2017a}; \citealt{deVis_2017b}).

For these reasons, studying the distribution of dust masses for a population of galaxies across time can provide important insights into the properties and evolution of galaxies and their ISM. The dust mass function (DMF), the space density of galaxies as a function of dust mass, is a fundamental measure of the dust content of galaxies, providing crucial information on galaxy evolution, star formation activity and the cosmic dust budget. In this study we use the results of the crossmatching analysis of the South Galactic Pole field of the \textit{Herschel}-ATLAS survey, as presented in the previous Chapter, to derive the far-IR selected DMF and investigate whether it has evolved with redshift from the local DMF to z $\sim$ 1.

\section{Local and High Redshift DMFs in the Literature}

Using surveys selected from sub-mm wavelengths provides a number of advantages when determining the DMF. At longer sub-mm wavelengths we sample the Rayleigh-Jeans tail of the Planck function where the flux density is least sensitive to the temperature of the dust and most sensitive to the dust's mass. Prior to the surveys conducted with \textit{Herschel}, \textit{Planck} and SCUBA in the local Universe, estimates of the local DMF often started from local \textit{IRAS} 60\,\micron luminosity functions (LF) and extrapolating this to sub-mm wavelengths on the assumption of an average FIR SED with a single dust temperature and dust emissivity index $\beta$ for all galaxies (see Chapter \ref{chapter:Dust_Evolution} for detail on the role of dust temperature and $\beta$ on the FIR SED). Naturally this lead to high dependence on the assumed dust temperature and $\beta$ value. Given that temperature evolves with luminosity (see our study on the temperature evolution with redshift for dusty star forming galaxies, DSFGs; Figure \ref{fig:wavepeak_lir}), extrapolating from one wavelength to another inherently contained a bias at high and low luminosity scales, causing a distortion in any DMF measured from the translation of such a luminosity function (\citealt{Dunne_2000}). The advent of \textit{Herschel} and \textit{Planck} allowed for large samples detected at far-IR wavelengths closer to the peak of the dust emission, enabling estimates of the DMF directly from sub-mm wavebands. The disadvantage of these long wavelength derived DMFs is the large beam size at these operating wavebands, causing blending and source misidentification (as illustrated through the use of the Likelihood Ratio method in the previous chapter).

The first direct measurements of the sub-mm derived DMF were with SCUBA using the SCUBA Local Universe Galaxy Survey (SLUGS: \citealt{Dunne_2000}; \citealt{Dunne_2001}; \citealt{Vlahakis_2005}), a survey of local \textit{IRAS}-selected galaxies. However, these studies were limited by small number statistics and were limited to very small redshifts. A high redshift (z = 2.5) DMF for comparison with the local SLUGS DMF was presented in \citealt{Dunne_2003}, suggesting that galaxies with the highest dust masses have an order of magnitude more dust than locally (assuming pure dust mass evolution and no evolution in the number density of the most massive galaxies). An obvious drawback of these studies is that even when extapolations of the \textit{IRAS} 60\,\micron LF are no longer required, the SLUGS studies depended on sub-mm observations of samples of galaxies selected in other wavebands. Improvements were made with the introduction of the Balloon-borne Large Aperture Submillimeter Telescope (BLAST; \citealt{Devlin_2009}), which allowed for the detection and derivation of the DMF from a sample of galaxies selected from wavelengths spanning the peak of the FIR background (the operating wavelengths of BLAST were 250, 350 and 500\,\micron, as it was designed as a precursor to the \textit{Herschel}-SPIRE instrument). \citealt{Eales_2009} used BLAST data to arrive at similar conclusions, that there is strong evolution in the LF in the three BLAST bands and in the DMF out to z = 1. The concurrence of the two suggesting that the evolution in the sub-mm luminosity of these galaxies is directly related to the increase in their dust reservoirs. However, this study was also limited by small number statistics ($\sim$ 100 sources).

The first study to measure the evolution in the DMF using the large H-ATLAS survey was \citealt{Dunne_2011}, using a 250\,\micron selected sample of 1867 sources from the SDP. This represented an order of magnitude larger than previous studies, allowing for a significant direct measurement of the density of galaxies as a function of dust mass to a redshift of 0.5. The main conclusions about the DMF between 0 < z < 0.5 from this study were that dust masses of the most massive galaxies decreased by a factor of 4 or 5 over the past 5 billion years and the integrated dust density evolves with redshift according to $\rho_{\textrm{dust}} \propto (1+z)^{4.5}$. The local density is estimated to be $\rho_{\textrm{dust}, (z=0)} = 9.8\times10^4$\,$M_\odot$ Mpc$^{-3}$. This work was developed further in \citealt{Beeston_2018} by deriving the local (z < 0.1) DMF for the largest sample of galaxies at the time of writing ($\sim$ 16,000 galaxies), utilizing the crossmatching between the H-ATLAS and the GAMA spectroscopic survey detailed in \citealt{Bourne_2016}. The sample size of \citealt{Beeston_2018} permitted dust masses as low as $\sim$ $10^4$\,$M_\odot$ and therefore extended the observed range of the DMF by at least an order of magnitude at the lowest masses compared to previous measurements. This yielded better constraints on the low mass end of the DMF which despite accounting for a negligible amount of the dust budget compared to the most massive galaxies, contribute substantially in number. There are a wide range of predictions for the slope of the low dust mass end of the DMF as this regime is constrained by sources that tend to be nearby and faint, and suffer from low numbers in flux-limited surveys such as H-ATLAS. With the size of the \citealt{Beeston_2018} sample the measurements of the low mass end of the DMF in this study are currently our best estimate. Despite differences in the measured low mass slope of \citealt{Dunne_2011} and \citealt{Beeston_2018}, both have local dust mass densities (DMD) in good agreement.

More recently, \citealt{Driver_2018} produced an extended DMF and measure of the DMD out to z = 5. This study was based on an optically-selected sample of approximately 570,000 galaxies from GAMA, G10-COSMOS (\citealt{Davies_2015}; \citealt{Andrews_2017}) and 3D-HST (\citealt{Brammer_2012}; \citealt{Momcheva_2016}). While \citealt{Dunne_2011} had observed a decline in the DMD in the highest redshift bin of their study (z = 0.4 -- 0.5), this was postulated to be the result of incompleteness. The wealth of data in \citealt{Driver_2018} enabled the peak in the DMD to be constrained closer to z $\sim$ 1, corresponding to a lookback time of approximately 8\,Gyr. This could mean that the DMD coincides (or trails behind by the order of 1 -- 2 Gyr) with the peak epoch of star formation which is known to peak at z $\sim$ 2 (\citealt{Cucciati_2012}; \citealt{Burgarella_2013}; \citealt{Madau_2014}). Unlike \citealt{Dunne_2011}, this study found no evidence for a strong evolution in the dust content of galaxies in the past 5\,Gyr, instead observing a flat DMD since z = 0.5. An apparent evolution in the dust luminosity observed by \citealt{Driver_2018} suggested that a strong evolution in dust temperature is required in order to maintain a flat DMD. Other notable works that predict the evolution of the DMF to high redshifts include \citealt{Pozzi_2020} and \citealt{Dudzeviciute_2021}. The former derived the DMF from z $\sim$ 0.2 up to z $\sim$ 2.5 using a 160\,\micron \textit{Herschel}-PACS selected catalogue of approximately 5,300 galaxies in the COSMOS field. In a juxtaposition with \citealt{Dunne_2011} and \citealt{Driver_2018}, they find a peak in the redshift evolution of the DMD in accordance with \citealt{Driver_2018}, but a positive trend at z < 0.5, in agreement with the work of \citealt{Dunne_2011}. The implication being that consistency among these studies may depend strongly on the selection of the sample, survey area or any assumptions that may be made during the measurement of the dust masses. We note that of the studies used for comparison in this work, the sample of \citealt{Pozzi_2020} is selected from the shortest observed frame wavelength (160\,\micron), which may in part explain the differences observed in the DMFs due to the strong influence of dust temperature (we explore this further in Section {\color{red} X}). Finally, \citealt{Dudzeviciute_2021} add valuable constraints on the DMF at z = 1 -- 2 and z = 3 -- 4 based on two samples selected at wavelengths corresponding to nearly indentical rest frame $\sim$ 180\,\micron populations. At z < 2 \citealt{Dudzeviciute_2021} study the dust properties of 121 SMGs from the 450\,\micron SCUBA-2 Ultra Deep Imaging EAO Survey (STUDIES: \citealt{Wang_2017}; \citealt{Chang_2018}; \citealt{Lim_2020b}; \citealt{Lim_2020c}) and compare these results to an 850\,\micron SMG sample from the ALMA/SCUBA-2 Ultra Deep Survey (AS2UDS: \citealt{Stach_2018}; \citealt{Stach_2019}, \citealt{Dudzeviciute_2020}). 

The sub-mm data used to derive the dust masses in this work come from the second data release of the \textit{Herschel}-ATLAS and the redshifts are taken from the VIKING counterparts with a high reliability in the previous chapter. The dust properties are calculated from the $\textit{Herschel}$ fluxes, and thus our estimates of the DMF are likely to follow the same trends as observed in the previous H-ATLAS studies of \citealt{Dunne_2011} and \citealt{Beeston_2018}. The work presented herein allows us to expand on these works by extending the $\textit{Herschel}$ predicted DMF to higher redshifts as a result of the near-IR crossmatching. In addition, we implement an error analysis that propagates errors in photometric redshifts through to the final DMF, allowing us to use a sample devoid of spectroscopic redshifts. The spectroscopic coverage of the SGP is much lower than the GAMA fields, but in this way we are able to use the complete SGP catalogue with the following criteria: i) sources must be classified as galaxies accordingly with the method in Section \ref{sec:star_galaxy_classifier}; ii) SGP galaxies have a near-IR counterpart with a reliability > 0.8; and iii) the near-IR counterpart has a photometric redshift in HELP. This corresponds to a sample of 81,895 galaxies, making it the most statistically robust measurement of the DMF made with the $\textit{Herschel}$-ATLAS. The effects of using photometric redshifts for all galaxies on the measured dust masses and the corresponding shape of the DMF are explored in the subsequent sections.

\section{Dust Properties of H-ATLAS Galaxies}

In order to measure the dust properties of the H-ATLAS galaxies in the rest frame over a wide range of redshifts, we require an understanding of the typical SED and any changes it may exhibit over time. Most importantly when measuring the dust mass from a given photometric point, we must k-correct our results to the same rest frame wavelength causing our results to be highly dependent on the assumed dust temperature and dust emissivity index. Although galaxies contain dust with a range of dust temperatures, previous studies have shown that most of the interstellar dust has a temperature of $\sim$ 20\,K (e.g. \citealt{Dunne_2001}; \citealt{Vlahakis_2005}; \citealt{Draine_2007}; \citealt{Boselli_2010}; \citealt{Smith_2012b}; \citealt{Smith_2013}). Nonetheless, dust close to sources of heating such as star forming regions and AGN radiate at rest frame wavelengths $\lesssim$ 100\,\micron and can influence the temperature measured from an isothermal dust model. Ideally the dust mass of a galaxy would be estimated using a mass-weighted temperature of the dust at the wavelength being used in the study. This would require fitting a model with multiple temperature components and weighting the temperatures of the dust by their mass. However, it has already been shown that the cold dust reservoir at $\sim$ 20\,K has a significantly larger mass (\citealt{Pearson_2013}) and the difference between the mass weighted dust temperature and the isothermal temperature is often not significant (e.g. \citealt{Clark_2015}). For example, if we assume that H-ATLAS galaxies are well approximated by the two temperature model of \citealt{Pearson_2013} (Section \ref{sec:phot_z_Herschel}; where $T_{\textrm{hot}}$ = 46.9\,K, $T_{\textrm{cold}}$ = 23.9\,K and $\alpha = M_c/M_h$ = 30.1), then the mass weighted temperature is given by $T_{\textrm{dust, weighted}} = (M_cT_c + M_hT_h)/(M_c + M_h) \approx$ 24.6\,K, which is only marginally warmer than the cold dust component. In addition, to constrain a hotter dust component at $\lambda_{\textrm{rest}} \lesssim$ 100\,\micron we would require additional shorter wavelength photometry such as the \textit{Herschel}-PACS data. As shown in Table \ref{tab:snr_fraction}, the percentage of galaxies in our reliable sample (R > 0.8) that have significant detections at the PACS wavelengths decreases rapidly with redshift to as low as $\sim$ 6 -- 8\% by z $\sim$ 0.3. With so few galaxies having PACS detections across our redshift range, the ability to adequately fit an SED to the \textit{Herschel}-SPIRE wavelengths alone is almost the same when considering a one or multiple component model, the benefit of an isothermal model being the reduction in model parameters that may otherwise be left unconstrained if multiple temperatures are assumed.

\begin{table}
    \centering
    \begin{tabular}{|p{3cm}|p{1.75cm}|p{1.75cm}|p{1.75cm}|p{1.75cm}|p{1.75cm}|}
        \hline
        Redshift Interval & 100\,\micron & 160\,\micron & 250\,\micron & 350\,\micron & 500\,\micron \\
         & [> 3$\sigma$] & [> 3$\sigma$] & [> 4$\sigma$] & [> 4$\sigma$] & [> 4$\sigma$] \\
        \hline
        \hline
        0 < z < 0.2 & 22.6 & 28.0 & 99.3 & 31.4 & 5.0 \\
        0.2 < z < 0.4 & 6.0 & 7.8 & 98.6 & 20.3 & 2.7 \\
        0.4 < z < 0.6 & 3.0 & 4.4 & 96.9 & 28.1 & 4.6 \\
        0.6 < z < 0.8 & 1.4 & 2.8 & 96.0 & 37.5 & 6.7 \\
        0.8 < z < 1 & 0.9 & 2.1 & 95.3 & 45.0 & 8.0 \\
        \hline
    \end{tabular}
    \caption{Caption.}
    \label{tab:snr_fraction}
\end{table}

[A single SED will be used in V calculations so we need a dust temperature that reflects the redshift bin.]

[But do I need to address this now? It is only because of the calculation of V later that we assume a fixed value of T.]
