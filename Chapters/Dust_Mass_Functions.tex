\textit{
Things implicitly assumed in the following that I will write in previous sections:
\begin{enumerate}
    \item The dust cycle
    \item A conclusion on what the full H-ATLAS survey provides
\end{enumerate}
}

\section{Introduction}

\section{Interstellar Dust as a Tracer of Cosmic Evolution}

Interstellar dust is responsible for obscuring approximately half of all starlight since the Big Bang (\citealt{Puget_1996}; \citealt{Fixsen_1998}; \citealt{Dole_2006}; \citealt{Driver_2016}). Assuming that young, massive stars are the main culprit for the heating of the interstellar dust and produce the UV photons that get absorbed and reradiated to far-IR wavelengths (\citealt{Kennicutt_1998}; \citealt{Calzetti_2007}; \citealt{Kennicutt_2009}), then measuring extragalactic dust masses is useful in our understanding of obscured star formation at different times in cosmic history. This naturally results from dust being a by-product of star formation. In order to study the evolution in the star formation rate density of the Universe, the most direct method is to measure the mass of galaxies' molecular gas reservoirs from which the stars form and estimate the star formation activity from the evolution of the cosmic molecular gas mass density. As detailed in the subsequent Chapter, molecular hydrogen, $H_2$ (the main component of the molecular gas), cannot be easily observed from the cold ISM ({\color{red} Typical temperature range}), whereas dust emission provides a suitable alternative at these temperatures. Using independent measures of the cosmic gas mass density from dust-traced gas mass (assuming some gas-to-dust mass ratio) allows us to constrain both the contribution of obscured star formation and the dust content of galaxies over time. In addition, metals are produced and expelled during supernovae and from stellar winds, with some fraction becoming bound to dust grains in the ISM. Here they may be ejected from the galaxy by starburst winds or become part of the next generation of stars. As a result, the total dust mass is a fundamental property of the ISM that can be used to trace the evolutionary stage of the galaxy, especially when used in conjunction with stellar and gas masses (e.g. \citealt{Cortese_2012}; \citealt{deVis_2017a}; \citealt{deVis_2017b}).

For these reasons, studying the distribution of dust masses for a population of galaxies across time can provide important insights into the properties and evolution of galaxies and their ISM. The dust mass function (DMF), the space density of galaxies as a function of dust mass, is a fundamental measure of the dust content of galaxies, providing crucial information on galaxy evolution, star formation activity and the cosmic dust budget. In this study we use the results of the crossmatching analysis of the South Galactic Pole field of the \textit{Herschel}-ATLAS survey, as presented in the previous Chapter, to derive the far-IR selected DMF and investigate whether it has evolved with redshift from the local DMF to z $\sim$ 1.

\section{Local and High Redshift DMFs in the Literature}

Using surveys selected from sub-mm wavelengths provides a number of advantages when determining the DMF. At longer sub-mm wavelengths we sample the Rayleigh-Jeans tail of the Planck function where the flux density is least sensitive to the temperature of the dust and most sensitive to the dust's mass. Prior to the surveys conducted with \textit{Herschel}, \textit{Planck} and SCUBA in the local Universe, estimates of the local DMF often started from local \textit{IRAS} 60\,\micron luminosity functions (LF) and extrapolating this to sub-mm wavelengths on the assumption of an average FIR SED with a single dust temperature and dust emissivity index $\beta$ for all galaxies (see Chapter \ref{chapter:Dust_Evolution} for detail on the role of dust temperature and $\beta$ on the FIR SED). Naturally this lead to high dependence on the assumed dust temperature and $\beta$ value. Given that temperature evolves with luminosity (see our study on the temperature evolution with redshift for dusty star forming galaxies, DSFGs; Figure \ref{fig:wavepeak_lir}), extrapolating from one wavelength to another inherently contained a bias at high and low luminosity scales, causing a distortion in any DMF measured from the translation of such a luminosity function (\citealt{Dunne_2000}). The advent of \textit{Herschel} and \textit{Planck} allowed for large samples detected at far-IR wavelengths closer to the peak of the dust emission, enabling estimates of the DMF directly from sub-mm wavebands. The disadvantage of these long wavelength derived DMFs is the large beam size at these operating wavebands, causing blending and source misidentification (as illustrated through the use of the Likelihood Ratio method in the previous chapter).

The first direct measurements of the sub-mm derived DMF were with SCUBA using the SCUBA Local Universe Galaxy Survey (SLUGS: \citealt{Dunne_2000}; \citealt{Dunne_2001}; \citealt{Vlahakis_2005}), a survey of local \textit{IRAS}-selected galaxies. However, these studies were limited by small number statistics and were limited to very small redshifts. A high redshift (z = 2.5) DMF for comparison with the local SLUGS DMF was presented in \citealt{Dunne_2003}, suggesting that galaxies with the highest dust masses have an order of magnitude more dust than locally (assuming pure dust mass evolution and no evolution in the number density of the most massive galaxies). An obvious drawback of these studies is that even when extapolations of the \textit{IRAS} 60\,\micron LF are no longer required, the SLUGS studies depended on sub-mm observations of samples of galaxies selected in other wavebands. Improvements were made with the introduction of the Balloon-borne Large Aperture Submillimeter Telescope (BLAST; \citealt{Devlin_2009}), which allowed for the detection and derivation of the DMF from a sample of galaxies selected from wavelengths spanning the peak of the FIR background (the operating wavelengths of BLAST were 250, 350 and 500\,\micron, as it was designed as a precursor to the \textit{Herschel}-SPIRE instrument).
