Interstellar dust, small solid grains that are ubiquitous to the interstellar medium (ISM), are essential in the formation of stars and the growth of galaxies. The physical effects dust has on a galaxy are wide ranging; from acting as the sites of molecule formation like $H_2$ and cooling the gas in the ISM to form dense molecular clouds, to being useful as a tracer of gas metallicity and thus its evolutionary state. In addition, the presence of dust can be inferred from the observational effects it has on the electromagnetic (EM) spectrum of galaxies. Depending on the star formation activity, the quantity of dust, and the sources of heating, the spectrum of a galaxy can reveal important properties that tell us how they have formed and evolved. These solid grains reprocess the starlight from young, massive stars and reradiate this energy at far-infrared and sub-millimeter wavelengths. Since 2009, this region of the EM spectrum has been observed extensively with the \textit{Herschel Space Observatory}, providing higher sensitivity and better angular resolution imaging of cold dusty regions of galaxies than had been seen previously. In this Thesis, we make use of large \textit{Herschel} (and to a lesser extent South Pole Telescope) surveys of individual dusty galaxies to explore the relationship between dust emission and the formation and evolution of active star forming galaxies. First, we present the third data release of one of these large \textit{Herschel} surveys, the \textit{Herschel}-ATLAS, and our contribution to the data products. Second, we use this sample to quantify the dust content of galaxies over the past $8$ billion years. Following this, we investigate whether dust takes the same physical and chemical properties at all cosmic epochs, by studying individual galaxies between $z = 2$ and $z = 6$. Finally, we use multiwavelength datasets that we reliably match to \textit{Herschel} galaxies via radio emission, in order to study their evolutionary state.