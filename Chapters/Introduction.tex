Until Edwin Hubble's measurement of the distances to M31 and M33 using Cepheid variable stars in the early twentieth century (\citealt{Hubble_1925}), many astronomers believed that the Milky Way encompassed all matter in the Universe. Observational data of the time meant that all extragalactic sources, appearing as small, hazy patches of light in the sky, were indistinguishable from clusters of stars, gas and dust that are part of our own Galaxy. Objects that were not immediately identifiable as stars were given the name \textit{nebulae} (Latin for 'clouds') which included Galactic sources as well as hitherto unknown extragalactic sources such as the \textit{Andromeda Nebula}. The consequence of this confusion is still evident in astronomy today in the naming convention used for certain catalogues, such as the Messier Catalogue, which consists of star clusters, nebulae and supernova remnants within the Galaxy as well as other galaxies, including \textit{Andromeda} (M31).

Since this initial discovery, the number of catalogued galaxies in the observable Universe has been ever increasing. Thanks to the finite speed of light, the history of star formation in the Universe can be observed directly from the light of distant galaxies as we look back time. Deep observations allow us to explore the evolution of galaxies from the early Universe to the galaxies we observe around us today. In particular, the deepest fields give astronomers the opportunity to look back at a time when galaxies were first forming. In 1995, the \textit{Hubble Space Telescope} was directed toward a small patch of sky covering only 1/30th the diameter of the full moon, for 10 consecutive days in order to capture a "keyhole" view of the Universe. The resulting image, known as the Hubble Deep Field (HDF, Figure \ref{fig:hubble_deep_field}), revealed a spectrum of galaxies with various morphologies and colours, despite the narrow field appearing to have nothing remarkable to the naked eye. The isotropic distribution of galaxies in all lines of sight suggests that this small sample of the total sky represents a typical distribution of galaxies from the early Universe to today. 

[3,000 galaxies of all different colours and types. Relates to our questions on galaxy evolution.]

\begin{figure}
    \centering
	\includegraphics[width=0.9\columnwidth]{Figures/hubble_deep_field.jpeg}
	\caption{{\color{red}Caption.}}
	\label{fig:hubble_deep_field}
\end{figure}
