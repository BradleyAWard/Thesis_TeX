Until Edwin Hubble's measurement of the distances to M31 and M33 using Cepheid variable stars in the early twentieth century (\citealt{Hubble_1925}), many astronomers believed that the Milky Way encompassed all matter in the Universe. Observational data of the time meant that all extragalactic sources, appearing as small, hazy patches of light in the sky, were indistinguishable from clusters of stars, gas and dust that are part of our own Galaxy. Objects that were not immediately identifiable as stars were given the name \textit{nebulae} (Latin for 'clouds') which included Galactic sources as well as hitherto unknown extragalactic sources such as the \textit{Andromeda Nebula}. The result of which is still evident in modern astronomy with the naming convention used for catalogues previously compiled by the likes of Messier and Hubble, such as the Messier Catalogue, consisting of Galactic star clusters, nebulae, supernova remnants and other galaxies, including Andromeda (M31).

Since this initial discovery, the number of galaxies contained in the observable Universe that have been catalogued has been ever increasing. By one prediction, the number of 