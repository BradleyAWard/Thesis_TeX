\section{Thesis Overview}

The aim of this Thesis was to use observations at far-IR and sub-mm wavelengths, predominantly from the \textit{Herschel Space Observatory}, to investigate the dust properties of active star forming galaxies and to understand the role that they play in the galaxy building epoch of the Universe. We recall the research questions underpinning the research in this Thesis, that were presented in Chapter \ref{chapter:Introduction}: How has the dust content of galaxies evolved to the present day? Are the dust properties of galaxies the same at all cosmic epochs? And in what evolutionary stage do we observe dust-enshrouded, IR-bright galaxies; do they provide the link to the massive systems we observe in the local Universe today? 

This work heavily relies on \textit{Herschel} selected samples in order to make progress answering these questions. Starting with the H-ATLAS project, for which we present a comprehensive data release of near-IR counterparts in the SGP, and then use to derive their dust masses and the evolution in the space density of dust over the past $8\,$Gyr. We also make use of HerBS (as well as the SPT-SZ survey), which represents a collection of some of the brightest IR galaxies known, allowing us to investigate their dust properties and determine if they are comparable to the local Universe. This is of particular importance to extragalactic studies at high redshifts where it is not uncommon to assume that the dust is non-evolving and is therefore the same as Galactic and local galaxy interstellar dust. Finally, we make use of the HerMES coverage of COSMOS to utilize the exceptional multiwavelength coverage in the field and gain insight into the evolutionary stage of high redshift Herschel galaxies.

Below we outline the key results obtained from this Thesis and present possible areas of research that would expand on this work.

\section{Key Results}

\section{Future Work}
\section{Concluding Remarks}
