\section{Introduction}
\section{Dust Properties of DSFGs}

Interstellar dust plays a crucial role in the formation of galaxies as dust grains are the site of molecule formation like molecular hydrogen, H2, the primary fuel for star formation (Kennicutt \& Evans 2012). H2 is the most abundant molecule in the Universe but is difficult to observe directly unless originating from energetic environments. Alternatives include observing less abundant molecules such as CO and using conversion factors to estimate the mass of molecular hydrogen, or observing dust emission to estimate dust masses (as in the previous chapter) and assuming gas-to-dust ratios (e.g. Saintonge+2013) to convert these into estimates for the total gas mass in high-redshift galaxies (e.g. Eales+2012; Scoville+2014). Such studies have shown that galaxies at high redshift contain a higher fraction of gas than galaxies today (Scoville+2016,2017; Tacconi+20??; Millard+20??), showing that direct observations of dust emission are useful in our understanding of how galaxies grow and evolve. It is important to note, however, that studies that make these links between dust emission and the evolution of galactic properties make the basic assumption that properties of the dust remain constant with redshift.

\section{Obtaining Redshifts from Molecular Lines}
\section{Sample Creation}
\subsection{South Pole Telescope DSFGs}
\subsection{The HerBS Sample}
\section{Far-Infrared and Sub-mm Colours}
\section{The Modified Blackbody Model}
\section{SED Fitting of DSFGs}
\subsection{Posterior Distributions}
\subsection{The Dust Emissivity Spectral Index - Dust Temperature Degeneracy}
\section{Accuracy of Dust Parameters}
\subsection{In-Out Simulations of SPT and HerBS Sources}
\subsection{Results of Simulations}
\section{The Properties of Dust in DSFGs}
\subsection{The Diversity of Dust Emissivity Spectral Indices and their Evolution with Redshift}
\subsection{The Diversity of Dust Temperatures and their Evolution with Redshift}
\subsection{Implications of Evolving Dust Properties between 2 < z < 6}
\section{Conclusions}

\listoftodos