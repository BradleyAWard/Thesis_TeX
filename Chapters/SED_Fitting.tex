\textit{
Things implicitly assumed in the following that I will write in previous sections:
\begin{enumerate}
    \item We have had to assume that dust is the same at all times
\end{enumerate}
}

\section{Introduction}

[Previous section looks at the quantity of dust across time, but this is assuming that the dust itself is the same at all epochs.]

\section{Dust Properties of DSFGs}

Interstellar dust plays a crucial role in the formation of galaxies as dust grains are the site of molecule formation like molecular hydrogen, $H_2$, the primary fuel for star formation (\citealt{Kennicutt_2012}). $H_2$ is the most abundant molecule in the Universe but is difficult to observe directly unless originating from energetic environments. Alternatives include observing less abundant molecules such as CO and using conversion factors to estimate the mass of molecular hydrogen, or observing dust emission to estimate dust masses (as in the previous chapter) and assuming gas-to-dust mass ratios (e.g. \citealt{Saintonge_2013}) to convert these into estimates for the total gas mass in high-redshift galaxies (e.g. \citealt{Magdis_2012}; \citealt{Eales_2012}; \citealt{Scoville_2014}; \citealt{Santini_2014}; \citealt{Genzel_2015}). Such studies have shown that galaxies at high redshift contain a higher fraction of gas than galaxies today (\citealt{Tacconi_2010}; \citealt{Scoville_2016}; \citealt{Scoville_2017}; \citealt{Millard_2020}), showing that direct observations of dust emission are useful in our understanding of how galaxies grow and evolve. It is important to note, however, that studies that make these links between dust emission and the evolution of galactic properties make the basic assumption that properties of the dust remain constant with redshift. In the following we investigate the possibility of evolution in dust itself by modelling the dust emission from a sample of high-redshift galaxies and measuring their dust properties over a large expanse of cosmic history.

Of particular importance to us is the dust emissivity spectral index, $\beta$, which controls the frequency dependence of the emissivity of dust grains per unit mass. By assuming, as is customary, that the optical depth of a galaxy can be approximated as a power law of the form $\tau \propto \nu^\beta$, we are implicity assuming that $\beta$ encodes within it information about the dust grain properties such as their chemical composition and their size and growth. The assumed value of $\beta$ for a galaxy can have significant consequences on the assumed absorption properties of the dust grains and consequently on fundamental properties of the ISM in the galaxy such as the total mass of dust (\citealt{Bianchi_2013}; \citealt{Clark_2016}).

Theoretical models for dust (e.g. \citealt{Draine_1984}; \citealt{Draine_2011}; \citealt{Kohler_2015}) predict $\beta$ values to range between approximately 1 -- 2 depending on the chemical composition of the dust grains. Adopting suitable fixed values of $\beta$ have been vital for estimates of the dust temperature and dust luminosity of galaxies in past studies, particularly for high-redshift sources that often lack constraints in the far-infrared (e.g. \todo[color=green]{Add references}). A nominal value of $\beta$ = 2 is common practice in this scenario as it mimics the emissivity of mixtures of amorphous silicates and graphites that well represent the optical properties of Galactic dust grains. However, recent studies have shown that the value of $\beta$ can take a wide variety of values among local galaxies and even among different regions within the same galaxy. For example, \citealt{Lamperti_2019} model the far-infrared dust SEDs of 192 nearby galaxies from the JCMT dust and gas In Nearby Galaxies Legacy Exploration (JINGLE) survey and observed a range of temperatures for the cold dust between 17 and 30\,K and dust emissivity spectral indices between 0.6 and 2.2. Within M31 (Andromeda) \citealt{Smith_2012}, \citealt{Draine_2014} and \citealt{Whitworth_2019} identified a decrease in $\beta$ with galactocentric radius, potentially a result of $\beta$ evolving to higher values when observed in denser regions of the ISM due to grain coagulation. A follow up study by \citealt{Athikkat-Eknath_2022} compared the average $\beta$ measured inside and outside molecular clouds within M31, and while there was no evidence to support the idea that $\beta$ varies due to dense molecular gas, the radial variation in $\beta$ remained present. At higher redshifts, where the far-infrared part of the spectrum is spatially unresolved, it is not possible to constrain the true value of a galaxy's $\beta$ but can be used to define an \textit{effective} $\beta$ value that represents the integrated dust properties over the whole galaxy. [...]\todo[color=orange]{Importance of having an effective $\beta$ value.}

\section{Obtaining Redshifts from Carbon Monoxide Lines}

In order to study the evolution of the dust properties of DSFGs we require robust redshifts to place their formation in cosmic history and to determine accurate measurements of fundamental properties. Obtaining robust ages of galaxies in the form of spectroscopically determined redshifts is hampered by the poor spatial resolution of single-dish observations, which worsens with increasing redshift. While DSFGs can be discovered at far-infrared and sub-mm wavelengths to high redshifts directly from the photometry of the sub-mm source, selecting those with distinctly red sub-mm colours in the \textit{Herschel} bands (e.g. \citealt{Dowell_2014}; \citealt{Ivison_2016}; \citealt{Donevski_2018}; \citealt{Duivenvoorden_2018}), careful consideration of the selection wavelength is required to make optimal use of the negative K-correction. For example, the \textit{Herschel}-SPIRE bands increasingly probe the peak of the dust SED at higher redshifts, making them less sensitive to unlensed DSFGs beyond z $\sim$ 2 -- 3. Additionally, the dust-obscured nature of this population of galaxies makes the identification of counterparts at other wavelengths where spectroscopically determined reshifts are readily available more difficult, which is compounded by the poor resolution and source confusion in the sub-mm (see the Likelihood Ratio analysis of Chapter \ref{chapter:Data_Release_3} for an example).

A more reliable method for robust redshifts is to follow up single-dish observations with inteferometric measurements with interferometers such as the Atacama Large Millimeter/submillimeter Array (ALMA). An even more direct way of obtaining redshfits while observing sources in the sub-mm/mm wavebands is to observe molecular emission lines which can be directly associated with the sub-mm emission without the need for intermediary steps with high-resolution imaging. Recent advancements in the possible bandwidths of instruments like ALMA and the Northern Extended Millimetre Array (NOEMA) have allowed for the ability to detect spectral lines (typically from CO or [CII]) that emanate unambiguously from the sub-mm source. CO is the second most abundant molecule in the Universe after $H_2$ and has rotational transitions that produce some of the brightest lines in the millimeter spectrum. The brightness of the CO lines result from the abundance of CO, the low excitation energy of the transitions and the wavelengths at which they occur coinciding with regions of the spectrum with high atmospheric transmission probed by ALMA. A secondary advantage of using molecular emission to determine spectroscopic redshifts is that they are independent of the photometry used to describe the dust SED and are therefore less prone to bias. In Figure \ref{fig:redshift_ladder} we show the coverage of CO line transitions as functions of the redshift and observed wavelength. We see that at [...] wavelengths, there is a non-uniform coverage of CO transitions, meaning that sources believed to be at a particular redshift may have multiple line detections, allowing for unambiguous constraints on the redshift of the galaxy, single line detections, which allows for some abiguity to the redshift solution, or no line detections. In the regions where no CO lines can be detected by a particular instrument, "redshift deserts" appear as breaks in the redshift distribution of galaxies.

\todo[color=red]{Complete figure and add caption}
\begin{figure}
	\centering
	\includegraphics[width=0.75\columnwidth]{Figures/redshift_ladder.pdf}
	\caption{{\color{red} Complete Figure}}
	\label{fig:redshift_ladder}
\end{figure}

In this work we study bright infrared sources detected by \textit{Herschel} and the South Pole Telescope (SPT; \citealt{Carlstrom_2011}) that have spectroscopically determined redshifts to test whether their measured dust properties have evolved from the early Universe to the peak epoch of star formation (between 2 $\lesssim z_{\textrm{spec}} \lesssim$ 6). We are also interested in identifying possible diversity in the dust properties of DSFGs and whether differing selection methods influence the type of dust that is observed. We therefore implicity assume in the following work that changes in the properties of the interstellar dust in galaxies can be directly inferred from variations in the spectral index, $\beta$ and the temperature of dust grains.

\section{Sample Creation}
\subsection{South Pole Telescope DSFGs}

A population of IR-bright SMGs selected at 1.4\,mm were obtained from the South Pole Telescope - Sunyarv-Zel'dovich (SPT-SZ) survey (\citealt{Vieira_2010}; \citealt{Mocanu_2013}; \citealt{Everett_2020}) which covers approximately 2500\,deg$^2$. The depth of this survey reaches $\sim$ 20\,mJy at 1.4\,mm, corresponding to sources detected at greater than 4.5\,$\sigma$ significance. We define our SPT sample as the 81 sources that have had synchrotron dominated systems removed (based on their 1.4\,mm to 2\,mm flux density ratios) and flux-limited to $S_{\textrm{870\,\micron}}$ > 25\,mJy. The high flux density cuts imply that most IR-bright galaxies would be too faint for the SPT sample without having been magnified from gravitational lensing. As shown by \citealt{Weiss_2013}, the average magnification of these sources are $\mu \sim$ 15 (corresponding to intrinsic flux densities of $S_{1.4\,\textrm{mm}}$ = 1 -- 3\,mJy), which is similar to the flux densities of unlensed sources identified from blank field surveys in the sub-mm/mm wavebands (e.g. \citealt{Coppin_2006}; \citealt{Pope_2006}; \citealt{Weiss_2009}) and are thus likely to be representative of this population albeit at higher observed redshifts.

During ALMA Cycle 0 \citealt{Weiss_2013} conducted a blind redshift survey for 26 SPT sources with ALMA's Band 3 receiver (2.6 -- 3.6\,mm). In total 44 line features were identified in the survey as emission lines of $^{12}$CO, $^{13}$CO, CI, H$_2$O and H$_2$O$^+$. The spectra of the 26 sources could be categorized according to the ambiguity of their estimated redshifts: 12 sources had spectra with multiple clear line features, from which a unique redshift solution can be found from the ALMA scans alone; 11 had a single line feature for which other spectroscopic or photometric measurements would be requied to constrain the redshift; and three sources for which no line features were observed. The same observing strategy was used during ALMA Cycle 1 by \citealt{Strandet_2016} to extend the redshift survey of \citealt{Weiss_2013} with an additional 15 sources observed in ALMA Band 3. For sources with a single CO line detection during Cycle 0, \citealt{Strandet_2016} present ALMA 1\,mm (Band 6) follow-up observations and, when still not satisfactory, follow-up observations were made with the First Light APEX Submillimetre Heterodyne receiver (FLASH; \citealt{Heyminck_2006}), the Swedish-ESO PI receiver (SEPIA; \citealt{Billade_2012}) and the Z-spec camera (\citealt{Naylor_2003}) onboard the Atacama Pathfinder Experiment (APEX) targeting CO and [CII] lines for those that remained unambiguous during Cycles 0 and 1. Lastly, \citealt{Reuter_2020} concluded the SPT redshift survey during ALMA Cycles 3, 4 and 7 by presenting spectra for the remaining 40 of the total 81 sources that had yet to be scanned. This resulted in 41 new spectroscopic redshifts. The culmination of these studies is a sample of 81 SPT-selected DSFGs  each with a spectroscopic redshift in the range 1.9 < $z_{\textrm{spec}}$ < 6.9; the median redshift of the sample being $z_{\textrm{median}}$ = 3.9$\pm$0.2 (\citealt{Reuter_2020}).

The SPT-DSFGs have photometric coverage that fully traces the SED peak and Rayleigh-Jeans (R-J) tail of the dust emission for all sources. Most of the SPT sample have coverage that spans at least observed wavelengths between 250\,\micron and 3\,mm. This photometry includes flux densities measured at 250-, 350- and 500\,\micron (\textit{Herschel}-SPIRE), 870\,\micron (APEX-LABOCA), 1.4- and 2\,mm (SPT) and 3\,mm (ALMA). For a subset of 65 sources, 100- and 160\,\micron flux densities were measured with \textit{Herschel}-PACS. The flux densities from \textit{Herschel} were estimated by fitting a Gaussian to the SPIRE counterparts and taking the peak as the flux density at 250-, 350- and 500\,\micron, or by using apertures of 7' and 10' to extract the flux density at 100- and 160\,\micron. The LABOCA, SPT and ALMA flux densities were taken to be the peak flux density of the point sources observed on the respective continuum maps. For the 3\,mm observations with ALMA, this was estimated from the continuum maps obtained during the blind redshift search. Given the range of redshifts, the SPT sources have a minimum photometric coverage between rest frame wavelengths 83\,\micron $\lesssim \lambda_{\textrm{rest}} \lesssim$ 380\,\micron meaning that the peak of the dust emission at $\sim$ 100\,\micron is always constrained and the Rayleigh-Jeans tail is well sampled (note that in Section [...]\todo[color=green]{Add section reference when written} we place further constraints on which sources are included in this study based on the number of photometric constraints available on the R-J tail). The complete set of photometric observations for the SPT DSFGs can be found in Appendix D of \citealt{Reuter_2020} and in Appendix \ref{app:SPT_DSFG_photometry} including the estimated magnifications of each source due to gravitational lensing as measured by \citealt{Spilker_2016}. In the cases where a magnification could not be measured, the average value of $\mu$ = 5.5 is used.

\subsection{The HerBS Sample}

A second sample used in this study comes from the \textit{Herschel} Bright Sources (HerBS; \citealt{Bakx_2018}) catalogue, a sample selected from the brightest high-redshift sources detected from the H-ATLAS project. Using the SED template derived by \citealt{Pearson_2013} to represent typical H-ATLAS sources (see Section \ref{sec:phot_z_Herschel}), \citealt{Bakx_2018} estimated the redshift of each source and selected those that have a measured photometric redshift > 2 and are observed at a 500\,\micron flux density > 80\,mJy. Initially the sample consisted of 223 sources, but having removed nearby galaxies and known blazars (\citealt{Negrello_2010}; \citealt{Lopez-Caniego_2013}), the HerBS sample is reduced to 209 SMGs. Presented with the catalogue are observations at 850\,\micron with the SCUBA-2 instrument (\citealt{Holland_2013}) on the James Clerk Maxwell Telescope (JCMT) for 203 sources. A detected source with SNR$_{850\,\micron}$ > 3 and offset from the \textit{Herschel} source < 10" is observed for 159 galaxies. Given that the sensitivity of \textit{Herschel}-SPIRE observations are poor beyond redshifts z $\sim$ 2 [...]\todo[color=orange]{Give some justification here}, the selection of galaxies in the HerBS sample is limited to very bright, rare sources. A large percentage of the sample are thus lensed ultraluminous IR galaxies (ULIRGS) with far-IR luminosities in the range $10^{12} L_\odot < L_{FIR} < 10^{13} L_\odot$ and hyperluminous IR galaxies (HyLIRGS) with $L_{FIR} > 10^{13} L_\odot$.

Spectroscopic redshifts are obtained for a subset of sources in the HerBS catalogue (selected from the H-ATLAS South Galactic Pole field) as part of the Bright Extragalactic ALMA Redshift Survey (BEARS: \citealt{Urquhart_2022}; \citealt{Bendo_2023}; \citealt{Hagimoto_2023}). \citealt{Urquhart_2022} targeted 85 sources for CO line emission with ALMA and presented spectroscopic measurements for 71 sources associated with 62 HerBS fields (the ALMA Band 4 images with angular resolution of $\sim$ 2" revealed that only half of the fields contained just a single source, while several contained two or more objects with similar spectroscopic redshifts - suggesting possible locations of physially associated galaxies or single sources being lensed by a foreground deflector). The BEARS spectral line survey started in ALMA Cycles 4 and 6 using the Band 3 receiver of the Atacama Compact Array (ACA) and continued during Cycle 7 in Bands 3 and 4 with the ALMA 12\,m Array. All sources with a spectroscopic redshift have Band 3 observations with the 12\,m Array, 74 of which also have Band 3 observations. The 11 remaining sources without Band 3 observations from the 12\,m Array depend on the ACA. As shown in Figure \ref{fig:redshift_ladder} \todo[color=orange]{Check that this is illustrated} the coverage of Bands 3 and 4 allows for the detection of CO lines between (2 -- 1) and (6 -- 5) depending on the redshift of the source; \citealt{Urquhart_2022} find that line detections are primarily found from the CO(3 -- 2), (4 -- 3) and (5 -- 4) transitions (38, 36 and 28 sources respectively). The frequencies of the spectral lines are transformed to redshift solutions using the graphical method described in \citealt{Bakx_2022} and, in doing so, find redshifts for 59 sources based on multiple spectral lines suggesting an unambiguous redshift. An additional 13 sources with emission from a single line have redshift solutions that can be relied upon since the CO emission is notably bright. Since adjacent CO lines typically have similar integrated line fluxes, alternative redshift solutions can be excluded when a single strong emission line is present. HerBS sources are retained in this study if they have spectroscopic redshifts determined from a single CO line.

In the cases where the HerBS source is deblended in the ALMA images we assume that the redshift of the group is the average spectroscopic redshift of all components, providing they are within 0.1 of each other. If there is only a single redshift corresponding to one of the components (and the redshift corresponding to the integrated emission of all sources is not provided), we assume that the redshift is applicable for the whole field. While \citealt{Bendo_2023} show that the brightest component (alphanumerically labelled A for each source) produces < 80\% of the total emission at 2\,mm, it is only in 4 fields (HerBS-56, -131, -138 and -146) that the spectroscopic redshift is assumed from components that does not include the brightest.

The photometry available for HerBS sources covers a similar range to the SPT sample: \textit{Herschel}-PACS (100-, 160\,\micron), \todo[color=orange]{Include these values} \textit{Herschel}-SPIRE (250-, 350-, 500\,\micron), SCUBA-2 (850\,\micron), ALMA (2-, 3\,mm). The ALMA flux densities were serendipitously estimated from the continuum of the spectrosopic survey. We also include snapshot continuum observations in ALMA Band 6 (1.1 -- 1.4\,mm) for a selection of ultra-red objects from the H-ATLAS that were taken as part of project 2018.1.00526.S (\textit{3000 dusty starbursts at z > 4}). For sources where the reduced angular resolution of the ALMA beam resolves the HerBS source into multiple components, \citealt{Bendo_2023} report the integrated flux densities of all objects if they lie within twice the FWHM of the ALMA beam, or if they are connected by structures that are themselves detected at greater than 3$\sigma$ significance. We preferentially chose to use the integrated fluxes for each HerBS source if available, or failing this, combine fluxes of individual components providing they are all detected. For example, the HerBS-63 field showed two detected ALMA sources in the Band 4 images, while a third source appeared in the Band 3 image misaligned with the other two. The third component (HerBS-63C) may not have been detected at 2\,mm because of poor sensitivity at the edge of the 3\,mm image or because it contains some contribution from AGN emission making it much brighter at longer wavelengths. In either case this inhibits us from knowing the integrated flux in both bands for all components, assuming that they may all be physically related, and so has been left out of the sample used here. In this sense we have taken a cautious approach to combining photometry. The shorter selection wavelength compared to the SPT sample leads to a narrower and lower redshift distribution; the minimum and maximum redshifts obtained by \citealt{Urquhart_2022} were $z_{\textrm{min}} = 1.407$ and $z_{\textrm{max}} = 4.509$. Considering the spectroscopic redshifts of the HerBS sources used in this study, the minimum photometric coverage is between rest frame wavelengths of $\sim$ 97\,\micron and $\sim$ 392\,\micron. The photometric coverage of HerBS galaxies has been tabulated in Appendix \ref{app:HerBS_photometry}.

\subsection{The Combined SPT and HerBS Samples}

In total there are 143 sources with a spectroscopic redshift across the two samples (81 SPT, 62 HerBS-BEARS). However, as we are interested in measuring the galaxy-integrated dust emissivity spectral index of each source, which is characterized by the slope of the emission on the R-J side of the Planck function, we make it a requirement of the final samples that there are at least two observations at observed wavelengths greater than 1\,mm for each galaxy. This has the effect of reducing the final group of galaxies studied here to 109 (79 SPT and 30 HerBS-BEARS). In the following we treat the two populations separately to highlight any potential differences that may result from the different selection wavelengths and flux limits. The redshift distribution is illustrated in Figure \ref{fig:spt_herbs_redshift}.

\begin{figure}
	\centering
	\includegraphics[width=0.75\columnwidth]{Figures/spt_herbs_redshift_distribution.pdf}
	\caption{The spectroscopic redshift distributions for the SPT-DSFG (grey) and HerBS (red) samples.}
	\label{fig:spt_herbs_redshift}
\end{figure}

\section{Far-Infrared and Sub-mm Colours}
\label{sec:fir_submm_colours}

A first estimate on the average dust properties of the two samples can be made by comparing their far-IR colours (which we define as the ratio between two flux densities in the FIR regime) with the predictions made using isothermal blackbody models. Previous studies have shown that FIR colours can be used as a proxy for dust properties (e.g. \citealt{Boselli_2010}; \citealt{Boselli_2012}; \citealt{Remy-Ruyer_2013}; \citealt{Smith_2019}) depending on the part of the FIR spectrum that the colour samples. For example, smaller wavelengths involving the \textit{Herschel}-PACS and \textit{Herschel}-SPIRE flux densities are more sensitive to changes in the dust temperature, while longer wavelengths are more sensitive to variations in the dust emissivity spectral index. In Figure \ref{fig:spt_herbs_colour_redshift} we show a selection of colours using flux densities between \textit{Herschel}-SPIRE 250\,\micron and ALMA 3\,mm that progressively travel across the spectrum from sampling the peak of thermal dust emission to the R-J tail. Using an isothermal modified blackbody model of the form $S_\nu \propto \frac{\nu^{\beta+3}}{e^{h \nu/kT} - 1}$, we predict the dependence of FIR colour on redshift for three combinations of dust temperature, $T_{\textrm{dust}}$, and $\beta$ ([T=30\,K, $\beta$=2], [T=30\,K, $\beta$=1.8] and [T=40\,K, $\beta$=2]). It can be seen that the assumed dust temperature has a significant effect on the location of the SED track for all FIR colours, while it is noticable that a change in $\beta$ has a larger effect on the observed colours at the longest wavelengths. 

\begin{figure}
    \centering
    \includegraphics[width=0.5\columnwidth,height=0.26\textheight]{Figures/spt_herbs_colour_250_500.pdf}
    \includegraphics[width=0.5\columnwidth,height=0.26\textheight]{Figures/spt_herbs_colour_350_2000.pdf}
    \includegraphics[width=0.5\columnwidth,height=0.26\textheight]{Figures/spt_herbs_colour_500_3000.pdf}
    \caption{Colour-redshift plots of $S_{\small 250\,\textrm{\textmu m}}$/$S_{\small 500\,\textrm{\textmu m}}$, $S_{\small 350\,\textrm{\textmu m}}$/$S_{\small 2\,\textrm{mm}}$ and $S_{\small 500\,\textrm{\textmu m}}$/$S_{\small 3\,\textrm{mm}}$ against spectrosopic redshift. The SPT sources are illustrated with grey circles and the HerBS sample as red circles. The predictions from three modified blackbody models assuming optical thin dust ([T = 30\,K, $\beta$=2], [T = 30\,K, $\beta$=1.8] and [T = 40\,K, $\beta$=2]) are plotted as solid, dashed and dotted black lines respectively.}
    \label{fig:spt_herbs_colour_redshift}
\end{figure}

Assuming for now that the dust emissivity index is well represented by $\beta \sim 2$ for all galaxies, then the FIR colours of most sources can be explained by a range of dust temperatures between T = 30 -- 40\,K assuming optically thin dust emission. There are notable exceptions, including a selection of SPT sources with high $S_{250\,\textrm{\micron}}/S_{500\,\textrm{\micron}}$ \todo[color=red]{Likely the 500um data, what does this imply?} and sources from both samples with high $S_{500\,\textrm{\micron}}/S_{3\,mm}$. At first this might suggest sources with large dust temperatures and/or high values of $\beta$, but other common culprits could be leading to these deviations from the bulk population. The most likely are the uncertainties on the flux density measurements or biases due to the choice of photometry. We note that our galaxies contain a combination of photometry obtained from single-dish observations (e.g. \textit{Herschel}, JCMT and SPT) and interferometers (ALMA). With their varying angular resolutions, emission observed with one instrument may be missed by another. In particular, we might expect that when ALMA resolves the single sources observed in the \textit{Herschel} wavebands into multiple components, some emission is lost giving the impression of higher FIR colours and thus higher dust temperatures and/or higher $\beta$.

In general, the SPT and HerBS galaxies occupy similar regions of the colour-redshift space, suggesting that the intrinsic dust properties of the two samples are likely to be similar. Given the scatter observed around the MBB models, it is appears unlikely that a single value of dust temperature and $\beta$ represents the observed SED of each galaxy equally well. \todo[color=red]{Description on how well the models represent the data - normalized distances should resemble a standardized normal distribution and colour-colour plots should be strongly correlated if the SEDs are well approximated by a single MBB with fixed beta.} 

The above results suggest that a single dust model assuming a constant $\beta$ value cannot explain the distribution of SEDs observed among the two populations. We are interested in ascertaining a single galaxy-integrated $\beta$, thus we must turn to fitting the observed SEDs individually to measure their dust properties. In the following we address two important questions about the properties of dust in high-redshift galaxies: i) is there diversity in the dust properties of high-redshift DSFGs? and ii) did these properties evolve over time? 

\section{SED Fitting of DSFGs}

We modelled the FIR to millimeter spectra by fitting a single MBB model combined with a mid-IR power law to all sources. In the FIR to mm regime the SED is dominated by a modified Planck function that represents the cold dust reservoir from which most of the mass of dust in the ISM is contained. This "cold" dust resides in the diffuse ISM and is heated by the ambient interstellar radiation field. However, a lower fraction of dust by mass radiates at hotter temperatures, heated by nearby star-forming regions, young OB stars or AGNs which contributes substantially to a galaxy's IR luminosity and dominates the emission at rest frame wavelengths $\lesssim$ 70\,\micron. This mid-IR component can be approximated using a power law of dust temperatures of the form $S_\nu \propto \nu^{-\alpha}$, where a lower value of $\alpha$ represents a higher fraction of emission emananting from sources other than the cold reservoir of dust, and higher values asymptotically tending towards a modified blackbody, representing a single temperature of dust.

The modified blackbody form is a direct result of the radiative transfer equation, $\frac{dI_\nu}{ds} = -\kappa_\nu \rho I_\nu + j_\nu \rho$, where $\kappa_\nu$ represents the opacity of the dust, $\rho$ is the density, $j_\nu$ represents the emissivity of the dust and $I_\nu$ is the spectral radiance (per unit area). If we define the source function, $S_\nu$ as $j_\nu/\kappa_\nu$ and the frequency dependent optical depth as $d\tau_\nu = -\kappa_\nu \rho ds$, then the radiative transfer equation becomes $\frac{dI_\nu}{ds} = I_\nu - S_\nu = I_\nu - B_\nu$, where we have used the fact that at local thermal equilibrium (LTE) the source function is equal to the Planck function, $B_\nu(T)$. The solution to the radiative transfer equation takes the form $I_\nu = (1 - e^{-\tau_\nu}) B_\nu(T)$. Given that the spectral radiance is proportional to the flux density at a given frequency, we can rewrite this as the modified blackbody model:

\begin{equation}
	S_{\nu, \textrm{obs}} = \Omega(1 - e^{-\tau_\nu}) B_\nu(T_{\textrm{dust}}),
\label{eq:modified_blackbody_omega}
\end{equation}

where $\Omega$ represents the solid angle subtended by the galaxy. Assuming the comoving distance between the source and observer is $D$, then the source solid angle is given by the ratio between the projected area of the source on the sky and the square of the distance. Therefore,

\begin{equation}
	S_{\nu, \textrm{obs}} = \frac{\mu A}{D^2}(1 - e^{-\tau_\nu}) B_\nu(T_{\textrm{dust}}),
\label{eq:modified_blackbody_area}
\end{equation}

where $A$ is the area of the source and we have included a multiplicative factor $\mu$ to account for the possibility of magnification due to gravitational lensing. For any galaxy where there is no evidence to suggest that gravitational lensing is magnifying the source, we may set $\mu = 1$. The comoving distance is related to the luminosity distance, $D_L$, by the equation $D = D_L/(1+z)$, providing we assume a flat Universe where $\Omega_k = 0$ (\citealt{Hogg_1999}):

\begin{equation}
	S_{\nu, \textrm{obs}} = \frac{\mu A (1+z)}{D_L^2}(1 - e^{-\tau_\nu}) B_\nu(T_{\textrm{dust}}).
	\label{eq:modified_blackbody_area_dl}
\end{equation}

As suggested above, the optical depth, $\tau_\nu$, is defined as the product of surface mass density, $\Sigma_{\textrm{dust}} = M_{\textrm{dust}}/A$ and the dust opacity, $\kappa_\nu$, but is often assumed to take the form of a power law, $(\nu/\nu_1)^\beta$, where $\nu_1$ represents the frequency at which the optical depth equals unity, and thus the transition between optically thick and optically thin media. The dust opacity is also generally described by a power law, $\kappa_\nu = \kappa_0(\nu/\nu_0)^\beta$, where $\kappa_0$ is the emissivity of the grains per unit mass at some reference frequency $\nu_0$. Hereafter we shall adopt $\kappa_0 = 0.077$\,m$^2$kg$^{-1}$ at $\nu_0 = 353$\,GHz ($\lambda_0 = 850$\,\micron) \todo[color=green]{Add references}. Following substitution we find that

\begin{equation}
	S_{\nu, \textrm{obs}} = \frac{\mu A (1+z)}{D_L^2}\Bigg(1 - e^{- \frac{M_{\textrm{dust}}\kappa_\nu}{A}}\Bigg) B_\nu(T_{\textrm{dust}}).
	\label{eq:modified_blackbody_general_opacity_a}
\end{equation}

The final amendment to the MBB models is to account for the heating of the dust due to the ambient temperature of the Cosmic Microwave Background (CMB). At high redshifts the CMB becomes a non-negligible source of heating, which unaccounted for in the model, could bias estimates of the dust temperature and dust emissivity index. When the CMB temperature at the redshift of the galaxy is a significant fraction of the cold dust temperature of the ISM within the galaxy, then we observe a change in the shape of the FIR SED (\citealt{daCunha_2013}). At local thermal equilibrium, the increase in the ISM temperature due to CMB heating at higher redshifts ($T_{\textrm{CMB}} = T_{\textrm{CMB}, 0}(1+z)$, where $T_{\textrm{CMB}, 0}$ is the temperature of the CMB today = 2.72\,K) has two competing effects on the observed dust emission. First, the dust continuum emission is boosted by the increased temperature of the CMB; second, the increased temperature creates a stronger background from which we observe the dust continuum. The net result on the SED is explained further in \citealt{daCunha_2013}. We account for the effects of dust heating by the CMB using the procedure explained therein. The CMB-adjusted general opacity blackbody model is now given by 

\begin{equation}
	S_{\nu, \textrm{obs}} = f_{\textrm{CMB}}\frac{\mu A (1+z)}{D_L^2}\Bigg(1 - e^{- \frac{M_{\textrm{dust}}\kappa_\nu}{A}}\Bigg) B_\nu(T_{\textrm{dust}}(z)),
		\label{eq:modified_blackbody_general_opacity_a_cmb}
\end{equation}

where we have made two changes. First, the prefactor $f_{\textrm{CMB}}$ denotes the fraction of the total dust emission that is observed against the background of the CMB and is given by Equation 18 of \citealt{daCunha_2013}; $f_{\textrm{CMB}} = \frac{S_\nu^{\textrm{observed}}}{S_\nu^{\textrm{intrinsic}}} = 1 - \frac{B_\nu[T_{\textrm{CMB}}(z)]}{B_\nu[T_{\textrm{dust}}(z)]}$. Second, we have redefined the dust temperature to be a function of redshift, $T_{\textrm{dust}}(z)$, and is given by Equation 12 of \citealt{daCunha_2013}; $T_{\textrm{dust}}(z) = [T_{\textrm{dust}, 0}^{4+\beta} + T_{\textrm{CMB}, 0}^{4+\beta} ((1+z)^{4+\beta} - 1)]^{\frac{1}{4+\beta}}$, where $T_{\textrm{dust}, 0}$ is the dust temperature at a redshift of zero. Note that in all future references of the dust temperature of a galaxy, we refer to the luminosity-weighted, CMB-corrected temperature as defined above unless otherwise stated.

In this general form of the modified blackbody there are up to four parameters describing the dust properties of the galaxy: the dust mass, $M_{\textrm{dust}}$, the characteristic dust temperature, $T_{\textrm{dust}}$, the radial size of the source, $r$, and the dust spectral index, $\beta$. The simplest approximation one can make is that the dust emission is optically thin, $\tau_\nu \ll 1$, which simplifies the self opacity term from $(1 - e^{-\tau_\nu})$ to $\tau_\nu$, and removes the necessity for defining the size of the continuum emission:

\begin{equation}
	S_{\nu, \textrm{obs}} = f_{\textrm{CMB}}\frac{\mu (1+z)}{D_L^2}M_{\textrm{dust}}\kappa_\nu B_\nu(T_{\textrm{dust}}(z)).
	\label{eq:modified_blackbody_optically_thin}
\end{equation}

For completeness, the sources were modelled using the optically thin approximation (Equation \ref{eq:modified_blackbody_optically_thin}) and in the general scenario where dust is allowed to remain optically thick at FIR wavelengths (Equation \ref{eq:modified_blackbody_general_opacity_a}). Given how high the dust masses are expected to be for the \textit{Herschel} and SPT sources, it is not unreasonable to expect that the dust might be optically thick at wavelengths probed by our photometry (e.g. \citealt{Conley_2011}; \citealt{Casey_2019}; \citealt{Cortzen_2020}). 

Some of the SPT sources have intrinsic size measurements from lens modelling in \citealt{Spilker_2016}, although caution should be exercised when using such estimates as the size is measured at a particular wavelength (in this case at 870\,\micron with ALMA imaging), and it is unlikely to be the same at all wavelengths due to differential lensing. However, we assume no differential lensing and take the effective radii as the size of the dust continuum region for the [...]\todo[color=green]{Add number} SPT galaxies modelled in \citealt{Spilker_2016}. Their average size is $\sim$ 1\,kpc. The remaining SPT and HerBS sources without known sizes require an alternative way of constraining the dust opacity. From the definitions of the optical depth given above we see that the transitional frequency is given by $\nu_1 = \nu_0(\kappa_0 \Sigma_{\textrm{dust}})^\beta$. Common values of $\lambda_1$ are 100\,\micron and 200\,\micron (\citealt{Blain_2003}; \citealt{Draine_2006}; \citealt{Conley_2011}; \citealt{Rangwala_2011}; \citealt{Greve_2012}; \citealt{Casey_2014}; \citealt{Spilker_2016}; \citealt{Casey_2019}; \citealt{Cooper_2022}; \citealt{Drew_2022}). For comparison with other works in the literature, we made no prior assumption about the value of $\lambda_1$ and modelled the SEDs of all sources assuming a value of 100\,\micron and 200\,\micron. In Section \ref{sec:comparison_optically_thin_and_general_opacity} we evaluate which approximation is most suitable by comparing the derived parameters for the SPT sources with lens models and with well studied sources in the literature that have had their dust opacities measured.

The lensing magnifications required to calculate the instrinsic properties of the sources come from the lens modelling in \citealt{Spilker_2016} for the SPT-DSFGs and from \citealt{Urquhart_2022} for the HerBS sample. The SPT-DSFGs range in magnification factors from 1 to 33, with a median value $\mu_{\textrm{median}} = 5.5$ which is assumed for all SPT sources not included in the study of \citealt{Spilker_2016}. The magnification factors of HerBS sources are derived in a different manner. It is well known that there is a correlation between the CO luminosities of submillimeter galaxies and their line widths (\citealt{Bothwell_2013}; \citealt{Dannerbauer_2017}; \citealt{Neri_2020}), in a CO analogy of the Tully-Fisher relationship (\citealt{Tully_1977}). The effect of gravitational lensing is to amplify the apparent CO luminosity, while the line width is unaffected. The offset from the relationship defined by a population of unlensed sources can then be used to estimate the extent to which a source is being lensed. In the case of the HerBS sources in this study, the magnification factors are in the range 1 -- 50 and the median is $\mu_{\textrm{median}}$ = 5.3.

During SED fitting we assumed flat priors on the dust mass between $10^5$ -- $10^9\,M_\odot$, the dust temperature between $T_{\textrm{CMB}}$ -- 100\,K, $\beta$ between 0.5 -- 6 and $\alpha$ between 0.5 -- 8. If a measurement of the intrinsic size of the source is available it is used during fitting, otherwise we assumed $\lambda_1$ = 100\,\micron and $\lambda_1$ = 200\,\micron in two separate applications of the general opacity model. The best fitting SEDs are determined from a Markov Chain Monte Carlo (MCMC) algorithm using the \texttt{emcee} package (\citealt{Foreman-Mackey_2013}) and assuming the general opacity model of Equation \ref{eq:modified_blackbody_general_opacity_a_cmb} and the optically thin model of Equation \ref{eq:modified_blackbody_optically_thin}. The dust properties are taken to be the median of their respective posterior distributions with 1$\sigma$ uncertainties quoted at the 16th and 84th percentiles. 

Calibration errors are added in quadrature with the flux density uncertainties during the fitting process according to an additional 7\% (\textit{Herschel}-PACS), 5.5\% (\textit{Herschel}-SPIRE), 5\% (SCUBA-2), 12\% (APEX-LABOCA), 7\% (SPT) and 10\% (ALMA). For consistency across the literature, we have conformed to the "best practices" outlined in \citealt{Drew_2022}. These recommendations aim to make cross-referencing comparisons between studies easier and provide a set of guidelines that allows the community to compare the dust properties of different galaxy populations coherently. In brief, these guidelines are: 

\begin{enumerate}
	\item Galaxies that lack photometric coverage should not be fit with models containing more free parameters than data. In the case of the MBB models used here, the maximum number of free parameters are five for the general opacity model ($M_{\textrm{dust}}$, $T_{\textrm{dust}}$, $\beta$, $\alpha$ and $r$/$\lambda_1$) and four when using the optically thin approximation ($M_{\textrm{dust}}$, $T_{\textrm{dust}}$, $\beta$ and $\alpha$), whereas there are no sources with fewer than six photometric constraints in the SPT-DSFG sample, and no sources with fewer than five in the HerBS sample.
	\item The number of free parameters should vary depending on the available photometric constraints. It is also recommended that Gaussian priors be used where data have low SNR and the parameters be fixed where data is limited, especially for $\lambda_1$ if there are no independent measurements for the dust column density. In this study we assume flat priors on all parameters given that we do not know the prior distribution and this will not be inferred through Hierarchical Bayesian fitting (see \citealt{Lamperti_2019} for an example of this method). However, in accordance with the best practices, we fix the wavelength where the dust opacity reaches unity either directly or indirectly from the galaxy's intrinsic size, given that we do not have substantial enough photometric coverage to constrain this parameter well.
	\item Poorly sampled SEDs that have no spatially-resolved dust continuum observations should focus on constraining the rest-frame peak wavelength $\lambda_{\textrm{peak}}$ rather than the dust temperature. As will be mentioned later, we do not consider the characteristic dust temperature obtained from our fits to be the true temperature of the dust, but rather a single, luminosity-weighted value that describes the bulk of the dust by mass in the ISM. Later we shall define an alternative dust temperature for the \textit{Herschel} and SPT galaxies as defined by their peak wavelength which is less biased by the choice of SED model.
\end{enumerate}

\section{Results of SED Fitting}

\subsection{Example SEDs: HerBS-11 and SPT0002-52}

In Figure \ref{fig:example_SEDs} we show the SED modelling of the first source alphanumerically in the two samples, HerBS-11 and SPT0002-52. Both sources do not have measurements for their intrinsic size and so two general opacity models are assumed, $\lambda_1$ = 100\,\micron and 200\,\micron. The SEDs of all other sources can be found in Appendices \ref{app:HerBS_SEDs} and \ref{app:SPT_DSFG_SEDs}. Our first look at how well the dust properties are constrained is to study the posterior distributions obtained from the converged MCMC chains of HerBS-11 and SPT0002-52. In the bottom panels of Figure \ref{fig:example_SEDs} we show the joint posterior distributions for the two sources. As was predicted from the FIR/sub-mm colours presented in Section \ref{sec:fir_submm_colours}, the two sources have similar posterior distributions that probe similar ranges in the parameter space. The most notable difference is that the parameter $\alpha$ is not constrained with any model for HerBS-11, which can be seen in the spread of acceptable SEDs as a result of having no \textit{Herschel}-PACS photometry. We note, however, that this has little effect on the dust properties of the cold ISM and is better than fitting an isothermal model to all datapoints which would skew the results to much higher average temperatures. 

\begin{figure}
	\centering
	\includegraphics[width=0.49\columnwidth]{Figures/spt_sed_example.pdf}
	\includegraphics[width=0.49\columnwidth]{Figures/herbs_sed_example.pdf}
	\includegraphics[width=0.49\columnwidth]{Figures/spt_example_contours.pdf}
	\includegraphics[width=0.49\columnwidth]{Figures/herbs_example_contours.pdf}
	\caption{Top: Example SED fits for SPT002-52 (black, left column) and HerBS-11 (red, right column). The best fitting optically thin and general opacity models, taken as the median value for each fitting parameter, are illustrated as solid (optically thin), dashed ($\lambda_1$ = 100\,\micron) and dotted ($\lambda_1$ = 200\,\micron) lines. The range of accepted SED fits are shown from 500 random draws from the posterior distribution, illustrated as the shaded regions. Bottom: The joint posterior distributions of HerBS-11 and SPT0002-52. The fitting parameters illustrated are the dust mass, the dust temperature and $\beta$. The derived parameters, IR luminosity (8 -- 1000\,\micron) and the peak wavelength, are also included. The three models are shown as increasingly lighter shades in the following order: optically thin, $\lambda_1$ = 100\,\micron and $\lambda_1$ = 200\,\micron.}
	\label{fig:example_SEDs}
\end{figure}

Common to both sources is the offset in some parameters depending on the opacity model that is being used. The optical depth and the dust temperature are strongly correlated such that an increase in the dust temperature and a decrease in the optical depth both shift the peak of the SED to shorter wavelengths. Then, given the relationship between the dust mass and dust temperature, \todo[color=orange]{provide relationship} the deviations of the optical depth from the truth leads to systematic offsets in the derived dust temperatures and dust masses. These systematics can be troubling for studies at high redshift where there is already tension about the mass of the ISM in dust and the modelling required to predict the ISM growth at such early epochs, otherwise known as the dust budget crisis (e.g. \citealt{Rowlands_2014}). On the other hand, some parameters appear to be well constrained and relatively indifferent to dust opacity. For example, the dust emissivity spectral indices show reasonable agreement given it is defined by the slope of the R-J tail which is relatively insensitive to the dust opacity and constrained by photometry which may lie comfortably within the optically thin approximation. To circumvent most of the uncertainty in the dust temperature ascribed to the choice of model, we preferentially adopted $\lambda_\textrm{peak}$ as a proxy for the average dust temperature of the dust, as this value is a fundamental property of the SED that is minimally affected by model assumptions. Note that while opacity assumptions are required to translate $\lambda_\textrm{peak}$ into a dust temperature, the peak wavelength itself may be used for comparisons between studies. Similarly, the IR luminosity, derived by integrating the dust SED between 8\,\micron and 1000\,\micron, is a fundamental quantity that is defined by the shape of the SED and thus not largely affected by our model assumptions. It is for these reasons that previous studies have used the integrated bolometric luminosity and the peak wavelength of galaxies' dust emission to assess whether dust temperatures and the DSFG contribution to the IR galaxy luminosity function (IRLF) evolved with redshift (e.g. \citealt{Casey_2018}; \citealt{Drew_2022}).

\subsection{Comparison between Optically Thin and General Opacity Models}
\label{sec:comparison_optically_thin_and_general_opacity}

While an individual galaxy's posterior distributions may not be particularly informative if the photometric constraints allow for a variety of SED models, the stacked posterior distributions of all sources can highlight preferred regions of parameter space and indicate the range and expected value for a given population of galaxies. Figure \ref{fig:stacked_posteriors} shows the stacked posterior distributions for all sources with relevant parameters shown with the lensing magnification included as this more closely resembles the set of direct observables. By not including the magnification factor we find a wider and less smooth posterior distribution for dust masses and IR luminosities. The median values of each parameter, averaged over the SPT and HerBS samples, were taken to be the median of the stacked posterior distribution with errors again quoted at the 16th and 84th percentiles. A summary of the average values for each parameter is given in Table \ref{tab:parameter_results}. The main parameter of interest, the dust emissivity speactral index, is well constrained at values of approximately 1.8 -- 2, which is in good agreement with the commonly accepted values for Galactic dust and nearby galaxies. Further discussion on this apparent lack of redshift evolution in $\beta$ is given in Section \ref{sec:redshift_evolution}. 

\begin{figure}
	\centering
	\includegraphics[width=\columnwidth]{Figures/stacked_posterior.pdf}
	\caption{The stacked posterior distributions of log($\mu M_{\textrm{dust}}$), $T_{\textrm{dust}}$, $\beta$, log($\mu L_{\textrm{IR}}$) and $\lambda_{\textrm{peak}}$ for SPT (top panels) and HerBS galaxies (bottom panels). The posterior distribution for each MBB model is illustrated as follows: optically thin (shaded), $\lambda_1$ = 100\,\micron (solid line), $\lambda_1$ = 200\,\micron (dashed line) and fixed continuum size (hatched). {\color{red} Remove padding on axis to make the figure central.}}
	\label{fig:stacked_posteriors}
\end{figure}

\begin{table}
    \centering
    \begin{tabular}{|p{3cm}|p{2.5cm}|p{2.5cm}|p{2.5cm}|}
        \hline
		Parameter & Optically Thin & $\lambda_1$ = 100\,\micron & $\lambda_1$ = 200\,\micron \\
		\hline
		\hline
		 & \multicolumn{3}{|c|}{SPT} \\
        \hline
        log($\mu M_{\textrm{dust}} [M_\odot]$) & $9.23_{-0.36}^{+0.40}$ & $9.08_{-0.41}^{+0.41}$ & $8.80_{-0.45}^{+0.47}$ \\
		$T_{\textrm{dust}}$ [K] & $31.94_{-8.30}^{+8.46}$ & $40.67_{-11.48}^{+10.92}$ & $55.67_{-9.31}^{+11.14}$ \\
		$\beta$ & $1.98_{-0.39}^{+0.54}$ & $1.91_{-0.32}^{+0.49}$ & $2.18_{-0.30}^{+0.38}$ \\
		log($\mu L_{\textrm{IR}} [L_\odot]$) & $13.99_{-0.31}^{+0.19}$ & $13.99_{-0.30}^{+0.19}$ & $13.88_{-0.26}^{+0.23}$ \\
		$\lambda_{\textrm{peak}}$ [\micron] & $90.97_{-12.80}^{+17.00}$ & $90.90_{-13.08}^{+17.79}$ & $89.22_{-13.38}^{+15.16}$ \\
		\hline
		\hline
		& \multicolumn{3}{|c|}{HerBS} \\
        \hline
        log($\mu M_{\textrm{dust}} [M_\odot]$) & $9.02_{-0.16}^{+0.28}$ & $8.90_{-0.18}^{+0.29}$ & $8.70_{-0.21}^{+0.31}$ \\
		$T_{\textrm{dust}}$ [K] & $32.72_{-5.86}^{+5.79}$ & $41.10_{-8.10}^{+7.64}$ & $55.15_{-9.22}^{+7.26}$ \\
		$\beta$ & $1.92_{-0.31}^{+0.38}$ & $1.85_{-0.25}^{+0.34}$ & $2.08_{-0.25}^{+0.26}$ \\
		log($\mu L_{\textrm{IR}} [L_\odot]$) & $13.54_{-0.37}^{+0.28}$ & $13.57_{-0.37}^{+0.26}$ & $13.58_{-0.34}^{+0.24}$ \\
		$\lambda_{\textrm{peak}}$ [\micron] & $89.96_{-9.10}^{+12.80}$ & $90.51_{-9.61}^{+12.96}$ & $89.96_{-9.46}^{+15.05}$ \\
        \hline
    \end{tabular}
    \caption{The median and 1$\sigma$ errors (estimated from the 14$^{th}$, 50$^{th}$ and 86$^{th}$ percentiles of the stacked posterior distribution) for the parameters presented in Figure \ref{fig:stacked_posteriors}.}
    \label{tab:parameter_results}
\end{table}

As predicted earlier, we observe systematic offsets in the median value of dust temperature and dust mass depending on the value of $\lambda_1$. For the optically thin MBB model we find median $T_{\textrm{dust}} = 31.9_{-8.3}^{+8.5}$\,K and $M_{\textrm{dust}} = 9.2_{-0.4}^{+0.4}$\,log($\mu M_\odot$) for SPT and $T_{\textrm{dust}} = 32.7_{-5.9}^{+5.8}$\,K and $M_{\textrm{dust}} = 9.0_{-0.2}^{+0.3}$\,log($\mu M_\odot$) for HerBS. There is a clear trend to higher dust temperatures and lower dust masses with increasing $\lambda_1$, with offsets of $\sim$ 10\,K and $\sim$ 20 -- 25\,K between the optically thin and $\lambda_1 = 100$\,\micron and $\lambda_1 = 200$\,\micron MBB models respectively. The dust masses show offsets of approximately 0.1\,log($\mu M_\odot$) and 0.4\,log($\mu M_\odot$) smaller for the general opacity models, though we note here that these are smaller than the 1$\sigma$ width of the distributions. There is no significant difference in the IR luminosities or peak wavelengths between any models, however, there are small offsets in the median $\beta$ values. While it is common to assume that $\beta$ is not affected by model assumptions due to minimal differences in the R-J slope, there are signs that at high enough opacities there is a small tendency towards higher $\beta$. This has been observed previously by \citealt{McKay_2023}. Using a sample of 870\,\micron-selected galaxies in GOODS-S, \citealt{McKay_2023} found good agreement in the median $\beta$ assuming an optically thin model (1.78) and a $\lambda_1 = 100$\,\micron MBB model (1.80), but this increased to 2.02 when assuming $\lambda_1 = 200$\,\micron. Recent studies have suggested that $\beta$ > 2 is common among high redshift galaxies (e.g. \citealt{Casey_2019}; \citealt{Casey_2021}; \citealt{Cooper_2022}), but this may be in part due to these studies assuming a higher wavelength where the dust remains optically thick. In this case, future studies concerning the evolution of the dust SED with redshift or comparing DSFGs in the literature, should consider whether $\lambda_1$ is consistent for all sources, and/or should have its own freedom to evolve.

Figure \ref{fig:stacked_posteriors} also illustrates the posterior distributions for the subset of SPT galaxies where we are able to model the area of the dust continuum region (hatched histogram). Providing that these galaxies do not represent a biased selection of the total sample, their dust properties are based on a self-consistent dust model that is useful for comparing with the remaining sources. In the dust properties where we observe the greatest discrepancy between models, the dust mass and dust temperature, we see that the fixed-$r$ MBB models tend to agree best with the optically thin and $\lambda_1 = 100$\,\micron models, suggesting $\lambda_1 = 200$\,\micron may not be appropriate for the majority of SPT galaxies. Using the same formalism described above, we can derive a value of $\lambda_1$ for each iteration of the MCMC and build a stacked posterior distribution of the derived parameter, $\lambda_1$. For the SPT sources with a fixed-$r$ model we find a median value of $\lambda_1 = 80.54_{-41.17}^{+46.75}$\,\micron. As we do not have radial size of any HerBS galaxies, we can not make the same comparison between posterior distributions and rely on the fact that the SPT and HerBS samples occupy similar regions of parameter space to validate the use of any of the general opacity models. A similar method would be to estimate the value of $\lambda_1$ from the SED fitting of well-studied galaxies in the literature that already have constraints on their emission area, providing they are analogous to the galaxies in the HerBS sample. 

{\color{red} [Compare with literature and provide $\lambda_1$ values]}

To determine whether the offsets in the assumed MBB models are equal for all sources, we compare the median likelihood estimates of each dust parameter derived using the optically thin assumption with those derived from the general opacity models, as shown in Figure \ref{fig:comparison_optically_thin_general_opacity}. While there is a strong correlation for all parameters, there is a clear trend in the dust mass and dust temperature such that the largest offsets occurs for galaxies with the least massive and hottest dust. {\color{red} Why?} A small deviation is observed in $\beta$ where for higher values the general opacity underpredicts the value when using the optically thin assumption, and the reverse is true at small values of $\beta$.

\begin{figure}
	\centering
	\includegraphics[height=0.9\textheight]{Figures/optically_thin_against_general_opacity.pdf}
	\caption{Comparison between the optically thin model and the general opacity models in predicting the dust properties of SPT (left column) and HerBS (right column) galaxies. The general opacity models are shown as filled and open circles for $\lambda_1$ = 100\,\micron and 200\,\micron respectively.}
	\label{fig:comparison_optically_thin_general_opacity}
\end{figure}

The optically thin, isothermal MBB model is widely used in the literature (e.g. \citealt{Magdis_2012}; \citealt{Simpson_2017}; \citealt{Lamperti_2019}; \citealt{Dudzeviciute_2020}; \citealt{Valentino_2020}; \citealt{daCunha_2021}) and provides a close match to the distribution observed with the fixed-$r$ model for our dust parameter of interest, $\beta$. In all subsequent analysis we shall consider the dust emission to be optically thin at all wavelengths, and reference the offsets described in this Section to approximate the results as if we had assumed a different opacity model. We note that this is likely to be the most consistent method as introducing an additional fixed parameter in $\lambda_1$, which is certainly going to vary among a population of galaxies, will introduce a bias in the results of each individual galaxy by an unknown amount depending on the true opacity of the dust emission. As mentioned earlier, allowing this parameter to vary will represent the most accurate representation of the dust emission, but will not be well constrained by photometry alone and therefore requires an independent measure of the dust column density for each galaxy.

\subsection{The $\beta$-Dust Temperature Degeneracy}

There is a known degeneracy between the dust temperature and $\beta$, which has been observed in a wide variety of scenarios in which the isothermal MBB has been used, from clouds in the Galaxy to galaxy-integrated $\beta$-$T_{\textrm{dust}}$ relationships from IR luminous galaxies (\citealt{Dupac_2003}; \citealt{Desert_2008}; \citealt{Paradis_2010}; \citealt{Schnee_2010}; \citealt{Veneziani_2010}; \citealt{Bracco_2011}; \citealt{Galametz_2012}; \citealt{Paladini_2012}; \citealt{Smith_2012}; \citealt{Lamperti_2019}; \citealt{daCunha_2021}).  While many studies have reported an anticorrelation between the dust emissivity index and dust temperature, it has also been shown that the strength of this correlation is dependent on the fitting process. An artificial anti-correlation can be introduced solely from large measurement uncertainties (\citealt{Shetty_2009a}; \citealt{Kelly_2012}; \citealt{Juvela_2012a}) and from assuming a constant temperature when the truth is that there are multiple dust temperatures along the line of sight (\citealt{Shetty_2009b}; \citealt{Juvela_2012b}).

Figure \ref{fig:beta_t_correlation} shows that there is a negative correlation between the measured values of $\beta$ and dust temperature for both samples when using the optically thin dust assumption. The strength of these correlations were tested with the Pearson correlation coefficient, $r_{\textrm{Pearson}}$, and Spearman's rank correlation coefficient, $\rho_{\textrm{Spearman}}$. The values of $r_{\textrm{Pearson}} = -0.83$ and $\rho_{\textrm{Spearman}} = -0.87$ for SPT and $r_{\textrm{Pearson}} = -0.89$ and $\rho_{\textrm{Spearman}} = -0.89$ for HerBS show that the SPT and HerBS galaxies exhibit a strong negative $\beta$-$T_{\textrm{dust}}$ correlation and the similarity between the two metrics suggests that few galaxies deviate from the trend. For accurate determination of the dust emissivity of our galaxies we must be sure that the correlation we observe is either a true property of the dust grains or that the influence of a degeneracy accounts for only a small spread in the posterior distribution of each $\beta$ value. In Figure \ref{fig:beta_t_correlation} we show the stacked joint posterior distribution between $\beta$ and dust temperature which highlights that the posterior closely follows the median likelihood values. If a strong degeneracy were causing an offset in our measured $\beta$ values then {\color{red} Explain this!}

\begin{figure}
	\centering
	\includegraphics[width=0.75\columnwidth]{Figures/beta_t_correlation.pdf}
	\caption{Caption. {\color{red} Add posteriors.}}
	\label{fig:beta_t_correlation}
\end{figure}

In the following section, we address the extent to which the observed $\beta-T_{\textrm{dust}}$ anti-correlation is a true relationship between dust properties, reflecting an intrinsic change in the emissivity properties of dust grains with temperature, and how much is a result of the fitting method. This is quantified by comparing our observed correlation with repeated SED fittings of simulated photometric data in which there is assumed to be no prior correlation between $\beta$ and $T_{\textrm{dust}}$.

\section{Simulations}
\label{sec:simulations}

In order to assess how accurately our fitting routine derives a galaxy's dust parameters, we ran a suite of mock SEDs with known input parameters and measured how precisely we recover the dust properties from our fits. There are three main aims of running input-output simulations of our SPT and HerBS galaxies: i) it provides a quantitative error on our dust parameters that tell us the scatter we might expect our dust properties to have from their true value; ii) we can test whether our main parameter of interest, $\beta$, is susceptible to variation due to the degeneracy with dust temperature. By providing a hypercube of parameters, any correlations observed in the output parameters can be attributed to the fitting process, which would otherwise be mistakenly taken as a physical correlation when applied to real galaxies; iii) the photometric bands sample the SEDs at the same observed wavelengths, corresponding to different rest frame wavelengths at different redshifts. The variation in sampling of the rest-frame SED with redshift could introduce biases in the measured dust properties, which would hinder our conclusions about the evolution of the average dust properties of DSFGs with redshift.

We generated the mock SEDs in the following way. First we must assume that the dust emission can be described by an isothermal blackbody in the optically thin regime (Equation \ref{eq:modified_blackbody_optically_thin}). A catalogue of {\color{red} X} models are produced with random parameters uniformly distributed between the lower and upper bounds presented in Table \ref{tab:simulation_inputs}, which are chosen to reflect the width of the posterior distributions observed for the real sources. To recreate mock galaxies with SEDs that reflect the properties of the HerBS and SPT samples the simulations were run twice. In the first instance each mock SED was placed at a random redshift between 2 and 4 and evaluated at the observed wavelengths of the HerBS sources. Note, however, that we did not include observations at 1.2\,mm as few sources in the final sample have photometric constraints at this wavelength, but all other wavelengths are included which results in a best scenario situation. Given the completeness of the \textit{Herschel} observations at all three SPIRE wavelengths and our requirement of having at least two constraints on the R-J tail at $\lambda_{\textrm{obs}}$ > 1\,mm, the inclusion of all wavelengths is a good approximation of our final HerBS sample. In a similar manner, in the second simulation, each iteration of input parameters is distributed between redshifts 2 and 6, reflecting the wider SPT redshift distribution, and the flux density measured at each observed wavelength of the SPT sample. Using the flux density errors for each source in the final sample, the range of SNR values was used to define a random error on each flux. Each flux density was then perturbed by a value generated from a normal distribution with a width equal to the mock galaxy's flux density error at that wavelength to represent the typical observational errors. For a reasonable scale of dust masses and far-IR luminosities, the median lensing magnifications of $\mu = 5.3$ and $5.5$ were used for all mock HerBS and SPT galaxies.

\begin{table}
    \centering
    \begin{tabular}{|p{3cm}|p{3cm}|p{3cm}|p{3cm}|p{3cm}|}
        \hline
        Parameter & Bounds \\
        \hline
        \hline
        log($\mu L_{\textrm{IR}} [L_{\odot}]$) & 13 -- 14 (HerBS) \\
        & 13 -- 14.5 (SPT) \\
		$T_{\textrm{dust}}$ [K] & 20 -- 50 \\
		$\beta$  & 0.5 -- 4 \\
		$\alpha$  & 1 -- 5 \\
        \hline
    \end{tabular}
    \caption{The upper and lower bounds assumed on the flat priors during the input-output simulations described in Section \ref{sec:simulations}.}
    \label{tab:simulation_inputs}
\end{table}

Each simulated galaxy carried with it a flag corresponding to whether it would have been included in their respective sample, having been perturbed by observational error. A source is deemed not detected if any of the perturbed flux densities falls below 29\,mJy at 250\,\micron or 80\,mJy at 500\,\micron for the HerBS simulation or below 25\,mJy at 870\,\micron or 16\,mJy at 1.4\,mm for the SPT simulation.

In Figure \ref{fig:in_out_simulations} we compare the derived dust properties to the input values for the two simulations, keeping only iterations where the true SED would have been detected by the corresponding survey. Given the wide coverage for SPT and HerBS SEDs, the fitting method recovers our dust properties with good accuracy. We see there are no systematic offsets with any of the fitting parameters. The root mean square error (RMSE) is calculated for each dust parameter showing the intrinsic scatter around the "true" input values. The RMSE indicate that our fitting recovers the input dust masses to {\color{red} 0.08}\,dex and {\color{red}0.05}\,dex, the dust temperatures to {\color{red} 2.61}\,K and {\color{red} 4.05}\,K and $\beta$ to {\color{red} 0.16} and {\color{red} 0.13} of the input values for SPT and HerBS galaxies respectively. In general the SPT galaxies with better sampling of their SEDs are marginally better constrained. The accuracy of the output $T_{\textrm{dust}}$ decreases for higher temperatures as a result of the SED shifting to shorter wavelengths, causing the peak to be less constrained by the \textit{Herschel} photometry than at lower temperatures.

\begin{figure}
	\centering
	\includegraphics[width=0.75\columnwidth]{Figures/in_out_simulations_herbs_ot_detected.pdf}
	\includegraphics[width=0.75\columnwidth]{Figures/in_out_simulations_spt_ot_detected.pdf}
	\caption{The input values compared to the measured output values for the simulations described in Section \ref{sec:simulations}. The panels illustrate the fitting parameters: dust mass (top left), dust temperature (top center) and dust emissivity index (top right), the derived parameters: IR luminosity (bottom left) and peak wavelength (bottom center), and the difference between the input and output dust temperature and $\beta$ (bottom right). The top two rows (red) show the results of the simulation of the optically thin HerBS galaxies while the bottom two rows (black) represent mock SPT galaxies.}
	\label{fig:in_out_simulations}
\end{figure}

The bottom right panel of the two simulations in Figure \ref{fig:in_out_simulations} show the difference between the input and output in the $T_{\textrm{dust}}$-$\beta$ plane. Given that the inputs of the simulations had no correlation between parameters, the anti-correlation observed in both simulations suggests that a degeneracy in the parameters appears solely from measurement uncertainties even if a physical correlation is not present in the data. Importantly, however, the difference between the input and output $T_{\textrm{dust}}$ and $\beta$ are centered on the origin, suggesting there is no tendency for over or underpredicted values.

\section{Possible Redshift Evolution in Dust Properties}
\label{sec:redshift_evolution}

In this section, we study the redshift evolution of the measured dust properties of HerBS and SPT galaxies in context with low redshift predictions from the Milky Way and local star forming galaxies. Given the ubiquitous use of local canonical $\beta$ values at higher redshifts, where other high redshift sources observed in the dust continuum typically have observations in fewer bands than the sources in this study, we can "re-calibrate" our canonical $\beta$ for high redshifts using these well sampled SEDs. We can also use the mock galaxies developed in the previous section to illustrate the effects of selection wavelength and detection limits on the expected evolution of dust properties and any trends that may be obfuscated by the bounds on parameter space set by future surveys. 

\subsection{Evolution of $\beta$ with Redshift}

Figure \ref{fig:beta_z_evolution} shows the distribution of galaxies in the $\beta$-z plane. To assess whether the range of measured $\beta$ values is a result of measurement uncertainties or represents a true diversity in the properties of interstellar dust, we make use of the simulated galaxies described in the previous section. On the assumption that the two samples are taken from two populations of galaxies that are composed of a single type of dust, we may expect that all galaxies from a particular sample would be consistent with a single value of $\beta$. Taking the median $\beta$ for each sample ({\color{red} X} and {\color{red} X} for SPT and HerBS respectively), we produce a further {\color{red} X} mock galaxies in the same manner as previously but with the median value being the input $\beta$ for all galaxies. 

%Taking the median $\beta$ and the typical error for each sample (1.98$\pm${\color{red} X} and 1.92$\pm${\color{red} X} for SPT and HerBS respectively), we restrict the simulations to include only those that have a randomly selected input $\beta$ within 1$\sigma$ of the median value. 

{\color{red} Complete simulations and $\beta$ section.}

\begin{figure}
	\centering
	\includegraphics[width=0.75\columnwidth]{Figures/beta_evolution.pdf}
	\caption{Caption.}
	\label{fig:beta_z_evolution}
\end{figure}

\subsection{Evolution of $T_{\textrm{dust}}$ with Redshift}

The nature of the redshift evolution of dust temperature is a matter of debate in the literature, at worst observational studies currently suggest entirely opposing trends. Advocating for hotter dust temperatures at high redshift (e.g. \citealt{Magdis_2012}; \citealt{Magnelli_2014}; \citealt{Swinbank_2014}; \citealt{Bethermin_2015}; \citealt{Faisst_2017}; \citealt{Schreiber_2018}; \citealt{Zavala_2018b}; \citealt{Liang_2019}; \citealt{Ma_2019}; \citealt{Faisst_2020}; \citealt{Bakx_2021}; \citealt{Witstok_2023}), such observational studies are backed by several possible scenarios. Higher specific star formation rates (sSFR), proportional to $L_{\textrm{IR}}/M_{\textrm{dust}}$, would lead to hotter temperatures, as would lower dust abundances and/or more compact dust emitting regions. Such studies attempting to fit the relation between dust temperature and redshift have struggled with the scatter observed at the highest redshifts, where there is disagreement about the use of a linear relationship or a plateau at z $\gtrsim$ 4. An important caveat of these observational studies is that there is a selection bias to higher luminosity as we move to higher redshifts, leading to a biased correlation with dust temperature. In contrast, the works of \citealt{Casey_2018}; \citealt{Jin_2019}, \citealt{Lim_2020a}, \citealt{Dudzeviciute_2020}, \citealt{Reuter_2020}, \citealt{Barger_2022}, \citealt{Drew_2022} and \citealt{Witstok_2023}, among others, report little or no evolution of temperature with redshift at a given fixed IR luminosity. Again, selection effects may obfuscate the true temperature evolution as we note that FIR selected samples will preferentially select for cold galaxies. As previously mentioned in \citealt{Bendo_2023}, the fundamental barrier to an unbiased relationship between dust temperature and redshift is the wavebands used to select the samples. Typically samples at FIR and sub-mm wavelengths select a higher proportion of atypical galaxies including particularly dusty galaxies and starbursts. Such galaxies diverge from the main sequence and may introduce scatter onto an otherwise tight temperature evolution one might expect from a survey of main sequence galaxies.

As mentioned earlier, the relationship between the characteristic dust temperature derived from the SED fitting and the peak wavelength are highly dependent on the adopted dust opacity model. While $T_{\textrm{dust}}$ and $\lambda_{\textrm{peak}}$ are both measures of the luminosity weighted temperature (a peak dust temperature can be obtained from $\lambda_{\textrm{peak}}$ using Wien's displacement law), only $\lambda_{\textrm{peak}}$ is directly constrained from the photometry. Neither form can be associated with a physical temperature of the dust in the ISM, but $T_{\textrm{peak}}$ is directly attributable to the shape of the SED. For the above reasons, we discuss evolution in dust temperature with redshift via the relation between $\lambda_{\textrm{peak}}$ and redshift at a fixed IR luminosity. 

The top panel of Figure \ref{fig:t_evolution} shows the distribution of the full sample of galaxies in IR luminosity as well as the detection limits assuming an SED with a dust temperature of 32\,K (reflecting the median temperature of the samples) and $\beta$ = 2. The 870\,\micron and 1.4\,mm limits in $L_{\textrm{IR}}$ with redshift follow the negative K-correction and thus provides a luminosity range that is accessible to SPT galaxies at all redshifts. The shorter wavelength \textit{Herschel}-SPIRE limits create a more complex selection function that is more sensitive to higher luminosity galaxies at higher redshift which becomes steeper at z $\sim$ 4 - though the strength of this bias and the redshift at which this break occurs depends inherently on the exact dust SED. The \textit{Herschel}-SPIRE limits cause an apparent deficit of lower luminosity sources ($\lesssim 5 \times 10^{13} L_{\odot}$, not accounting for lensing magnification) at z $\gtrsim$ 3. To model the evolutionary trend in dust temperature while controlling for the bias towards more luminous sources at higher redshifts, we consider sources selected in a small range of FIR luminosity, as illustrated by the boxed regions. The bottom panel of Figure \ref{fig:t_evolution} shows the peak dust temperature ($T_{\textrm{peak}} = 2.898 \times 10^{3}$ [\micron K]/$\lambda_{\textrm{peak}}$ [\micron]) for these sources compared to the observational trends inferred by \citealt{Schreiber_2018}, \citealt{Bouwens_2020} and \citealt{Viero_2022}, the median dust temperature of the JINGLE survey and the evolution of simulated galaxies by \citealt{Liang_2019}. Bootstrap fitting of the two samples with a linear model gives an evolution rate of 5.23$\pm$0.97 and -0.20$\pm$0.50 $T_{\textrm{peak}}$/z for the HerBS and SPT galaxies respectively. We observe no evidence for SPT galaxies having an increased dust temperature at higher redshifts, confirming previous evolutionary trends that cover the full luminosity range (e.g. \citealt{Reuter_2020}; \citealt{Witstok_2023}). While the HerBS galaxies suggest that dust temperature increases with redshift, we cannot be certain that the highest luminosity sources in the sample are free from bias at z $\lesssim$ 3 due to the \textit{Herschel} selection criteria. 

\begin{figure}
	\centering
	\includegraphics[width=0.75\columnwidth]{Figures/t_evolution.pdf}
	\caption{Caption.}
	\label{fig:t_evolution}
\end{figure}

\section{Conclusions}