\section{Introduction}

[In the previous chapter (DMFs) I show the evolution in dust masses assuming Galactic dust at all redshifts (i.e T=20K and beta=2), this chapter will tackle this assumption.]

\section{Dust Properties of DSFGs}

Interstellar dust plays a crucial role in the formation of galaxies as dust grains are the site of molecule formation like molecular hydrogen, $H_2$, the primary fuel for star formation (\citealt{Kennicutt_2012}). $H_2$ is the most abundant molecule in the Universe but is difficult to observe directly unless originating from energetic environments. Alternatives include observing less abundant molecules such as CO and using conversion factors to estimate the mass of molecular hydrogen, or observing dust emission to estimate dust masses (as in the previous chapter) and assuming gas-to-dust ratios (e.g. \citealt{Saintonge_2013}) to convert these into estimates for the total gas mass in high-redshift galaxies (e.g. \citealt{Magdis_2012}; \citealt{Eales_2012}; \citealt{Scoville_2014}; \citealt{Santini_2014}; \citealt{Genzel_2015}). Such studies have shown that galaxies at high redshift contain a higher fraction of gas than galaxies today (\citealt{Tacconi_2010}; \citealt{Scoville_2016}; \citealt{Scoville_2017}; \citealt{Millard_2020}), showing that direct observations of dust emission are useful in our understanding of how galaxies grow and evolve. It is important to note, however, that studies that make these links between dust emission and the evolution of galactic properties make the basic assumption that properties of the dust remain constant with redshift. In the following we investigate the possibility of evolution in dust itself by modelling the dust emission from a sample of high-redshift galaxies and inferring their dust properties over a large expanse of cosmic history.

Of particular importance to us is the dust emissivity spectral index, $\beta$, which controls the frequency dependence of the emissivity of dust grains per unit mass. By assuming as is customary that the optical depth of a galaxy can be approximated as a power law of the form $\tau \propto \nu^\beta$, we are implicity assuming that $\beta$ encodes within it information about the dust grain properties such as their chemical composition and their size and growth. The assumed value of $\beta$ for a galaxy can have significant consequences on the assumed absorption properties of the dust grains and consequently on fundamental properties of the ISM in the galaxy such as the total mass of dust (\citealt{Bianchi_2013}; \citealt{Clark_2016}).

Theoretical models for dust (e.g. \citealt{Draine_1984}; \citealt{Draine_2011}; \citealt{Kohler_2015}) predict $\beta$ values to range between approximately 1 -- 2 depending on the chemical composition of the dust grains. Adopting suitable fixed values of $\beta$ have been vital for estimates of the dust temperature and dust luminosity of galaxies in past studies, particularly for high-redshift sources that often lack constraints in the far-infrared (e.g. \todo[color=green]{Add references}). A nominal value of $\beta$ = 2 is common practice in this scenario as it mimics the emissivity of mixtures of amorphous silicates and graphites that well represent the optical properties of Galactic dust grains. However, recent studies have shown that the value of $\beta$ can take a wide variety of values among local galaxies and even among different regions within the same galaxy. For example, \citealt{Lamperti_2019} model the far-infrared dust SEDs of 192 nearby galaxies from the JCMT dust and gas In Nearby Galaxies Legacy Exploration (JINGLE) survey and observed a range of temperatures for the cold dust between 17 and 30\,K and dust emissivity spectral indices between 0.6 and 2.2. Within M31 (Andromeda) \citealt{Smith_2012}, \citealt{Draine_2014} and \citealt{Whitworth_2019} identified a decrease in $\beta$ with galactocentric radius, potentially a result of $\beta$ evolving to higher values when observed in denser regions of the ISM due to grain coagulation. A follow up study by \citealt{Athikkat-Eknath_2022} compared the average $\beta$ measured inside and outside molcular clouds within M31, and while there was no evidence to support the idea that $\beta$ varies due to dense molecular gas, the radial variation in $\beta$ remained present. 

\section{Obtaining Redshifts from Molecular Lines}
\section{Sample Creation}
\subsection{South Pole Telescope DSFGs}
\subsection{The HerBS Sample}
\section{Far-Infrared and Sub-mm Colours}
\section{The Modified Blackbody Model}
\section{SED Fitting of DSFGs}
\subsection{Posterior Distributions}
\subsection{The Dust Emissivity Spectral Index - Dust Temperature Degeneracy}
\section{Accuracy of Dust Parameters}
\subsection{Simulations of SPT and HerBS Sources}
\subsection{Results of Simulations}
\section{The Properties of Dust in DSFGs}
\subsection{The Diversity of Dust Emissivity Spectral Indices and their Evolution with Redshift}
\subsection{The Diversity of Dust Temperatures and their Evolution with Redshift}
\subsection{Implications of Evolving Dust Properties between 2 < z < 6}
\section{Conclusions}

\listoftodos