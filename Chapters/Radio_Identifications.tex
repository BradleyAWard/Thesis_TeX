\chapquote{"Video killed the radio star."}{The Buggles}{1979}

\section{Introduction}

\section{The Far-Infrared/Radio Correlation}

For samples of star forming galaxies in the local and high redshift universe there is a well observed correlation between the far-infrared and radio emission (e.g. \citealt{Dickey_1984}; \citealt{deJong_1985}; \citealt{Helou_1985}; \citealt{Condon_1992}; \citealt{Barger_2000}; \citealt{Yun_2001}; \citealt{Garrett_2002}; \citealt{Appleton_2004}; \citealt{Ibar_2008}; \citealt{Seymour_2009}; \citealt{Sargent_2010}), that remains nearly linear over multiple orders of magnitude in far-infrared luminosity ($10^{9} \lesssim L_{\textrm{FIR}} [L_{\odot}] \lesssim 10^{12.5}$). The small scatter in this relation is often attributed to the "calorimeter model" (\citealt{Voelk_1989}; \citealt{Lisenfeld_1996}; \citealt{Lacki_2010}), which ascribes the far-infrared and radio emission to common stellar sources. In this model, galaxies are assumed to be opaque to UV light from massive OB stars which gets absorbed by the dust in the ISM and reradiated in the far-infrared. Observations in the far-infrared and sub-mm are thus sensitive to the cold dust that reradiates the energy from young stars. These stars quickly come to the end of their lives, exploding as Type II supernovae producing cosmic ray electrons and positrons. Most of their energy gets radiated in the radio as synchrotron radiation when interacting with the magnetic fields of the supernova remnant. The radio continuum is therefore a tracer of recent, obscured star formation.

Despite a higher scatter for high redshift SMGs than with local samples, the far-infrared/radio correlation (FIRC) still holds at increasing redshift (\citealt{Ivison_2010a}; \citealt{Ivison_2010b}). {\color{red}Small description on evolution of FIRC with redshift.}

The appearance of a correlation at all redshifts and luminosities has facilitated studies that use the radio emission as an unbiased tracer of obscured star formation in dusty galaxies (\citealt{Kennicutt_2012}). By making use of the tight correlation (as well as the benefits we outline in the following section), we can identify counterparts to single dish sub-mm observations with large beams with greater confidence than we would expect from the optical or near-infrared. This allows for better characterization of the properties of SMGs and allows us to study the cosmic star formation history to higher redshifts (\citealt{Madau_2014}; \citealt{Delhaize_2017}; \citealt{Novak_2017}).

\section{Identifying Multiwavelength Counterparts to SMGs via Radio IDs}

As illustrated in Chapter \ref{chapter:Data_Release_3}, SMGs that are detected with single dish sub-mm telescopes with low angular resolution will likely include multiple galaxies in the optical or near-infrared within the beam width. This is a particular problem for \textit{Herschel} where even the smallest beam size at 250\,\micron\ has a FWHM of $\sim$ 18 arcsec. The possibility of tightly clustered SMGs (\citealt{Blain_2004}) and with a higher surface density of optical counterparts, deciding unambiguously on the source of the sub-mm emission is difficult. This is further compounded by their intrinsic faintness due to the dust obscuration. The optimal way to overcome the problem of poor resolution is to follow up with millimetric or longer wavelength interferometric observations. In the following sections we detail a method for identifying radio counterparts to \textit{Herschel} sources in order to secure unambiguous multiwavelength counterparts to our sub-mm galaxies. The following is a list of benefits of using the radio emission to select counterparts throughout the electromagnetic spectrum.

\begin{enumerate}
    \item Star forming galaxies produce a lot of synchrotron emission. This allows us to take advantage of the FIRC to locate the galaxy emitting radiation in the far-IR. Compared to searches in other wavelengths, such as using the Likelihood Ratio to identify optical counterparts, the search for identifications is not solely motivated by their position and brightness, but also by the physical link between the two objects due to their common source.
    \item Even when considering the deepest radio maps, the low surface density of radio sources means that the probability of chance positoinal coincidences with a sub-mm source is relatively unlikely. As a result, we can have a reasonable confidence in some association between objects when we do observe radio sources in close proximity to the sub-mm position (\citealt{Ivison_2002}; \citealt{Borys_2004}). In most cases (as we shall validate in this study), radio sources are sufficiently rare that finding an object within the positional uncertainty of the sub-mm beam almost always results in a robust identification.
    \item Reason 3
    \item Reason 4
\end{enumerate}


