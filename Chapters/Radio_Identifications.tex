\chapquote{"Video killed the radio star."}{The Buggles}{1979}

\section{Introduction}

\section{The Far-Infrared/Radio Correlation}

For samples of star forming galaxies in the local and high redshift universe there is a well observed correlation between the far-infrared and radio emission (e.g. \citealt{Dickey_1984}; \citealt{deJong_1985}; \citealt{Helou_1985}; \citealt{Condon_1992}; \citealt{Barger_2000}; \citealt{Yun_2001}; \citealt{Garrett_2002}; \citealt{Appleton_2004}; \citealt{Ibar_2008}; \citealt{Seymour_2009}; \citealt{Ivison_2010a}; \citealt{Ivison_2010b}; \citealt{Sargent_2010} {\color{red}More recent works}), that remains nearly linear over multiple orders of magnitude in far-IR luminosity ($10^{9} \lesssim L_{\textrm{FIR}} [L_{\odot}] \lesssim 10^{12.5}$). The small scatter in this relation is often attributed to the "calorimeter model" (\citealt{Voelk_1989}; \citealt{Lisenfeld_1996}; \citealt{Lacki_2010}), which ascribes the far-infrared and radio emission to common stellar sources. In this model, 


