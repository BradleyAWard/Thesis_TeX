\section{The Herschel-ATLAS}
\label{sec:The Herschel-ATLAS}

The \textit{Herschel} Astrophysical Terahertz Large Area Survey (H-ATLAS; \citealt{Eales_2010}) was the largest open-time sub-mm survey carried out with \textit{Herschel}. The survey was observed across five photometric bands using two instruments onboard the \textit{Herschel Space Observatory}: the Photodetector Array Camera (PACS, \citealt{Poglitsch_2010}) at 100 and 160\,\micron, and the Spectral and Photometric Imaging Receiver (SPIRE, \citealt{Griffin_2010}) at 250, 350 and 500\,\micron. Compared to the first SMGs detected using SCUBA at 850\,\micron (\citealt{Smail_1997}; \citealt{Barger_1998}; \citealt{Hughes_1998}), the PACS and SPIRE wavebands span the peak of the infrared spectrum for low redshift (z < 1) galaxies. Their intrinsic brightness at the SPIRE wavelengths makes their detection in the thousands more achievable. 

%{\color{orange}The main scientific goal of the H-ATLAS was to provide measurements of the dust masses and dust obscured star formation for tens of thousands of nearby galaxies (to create and FIR/sub-mm analogue to the SDSS). The original goal was to provide a shallow survey over a large area of sky, but the exceptional sensitivity of \textit{Herschel} and the negative k-correction at sub-mm wavelengths (References) means a significant fraction of sources lie at high redshifts (References). While the original goal was to study dust and newly formed stars hidden by dust in nearby (z < 0.4) galaxies, the negative k-correction mean that the median redshift of sources is z $\sim$ 1. The source catalogues include sources up to a redshift of at least 6 (References).}

The complete survey covers $\sim$\,660\,$\deg^2$, split into three regions located to avoid emission from Galactic dust and to utilize complimentary spectroscopic surveys including the Sloan Digital Sky Survey (SDSS, \citealt{York_2000}), the 2df Galaxy Redshift Survey (2dfGRS, \citealt{Colless_2001}) and the Galaxy and Mass Assembly (GAMA, \citealt{Driver_2009}). The North Galactic Pole (NGP) region covers $\sim$\,180\,$\deg^2$ of the northern sky, centered at R.A 13$^{h}$18$^{m}$ and declination +29$^{\circ}$13' (J2000); three equatorial fields, located at approximately R.A 9$^{h}$, 12$^{h}$ and 15$^{h}$ coinciding with the GAMA survey (henceforth named GAMA9, GAMA12 and GAMA15 fields), each with an area of approximately 54\,$\deg^2$, and the South Galactic Pole (SGP) region, centered at R.A 0$^{h}$6$^{m}$ and declination -32$^{\circ}$44' (J2000) with an area of $\sim$ 318\,$\deg^2$. 

\subsection{Detecting Submillimeter Sources on Herschel Images}
\label{sec:Detecting Submillimeter Sources on Herschel Images}

Due to {\color{orange}[...]} sub-mm images suffer from two types of noise; instrumental noise {\color{orange}[...]} and confusion noise which is highly correlated between pixels, most of its contribution coming from the blending together of faint sources. Source confusion is of particular importance to sub-mm surveys {\color{orange}[...]}. The result of combining instrumental noise with confusion noise is that almost all sources in the Herschel images are unresolved and the optimum filter for detecting these unresolved sources is no longer the point spread function (PSF). Consider a \textit{Herschel} map in which there is only one source of noise: an image with instrumental noise but no confusion noise (i.e. there is only one point source and no fainter, confusing sources), the optimal detection of this source is obtained by convolving the image with the PSF of the instrument. On the other hand, a map with no instrumental noise, but many confused point sources would be optimally detected with its best signal to noise ratio (SNR) by taking the Fourier transform of the image, dividing by the Fourier transform of the PSF and taking the inverse Fourier transform to obtain a perfect deconvolution of the original map (\citealt{Valiante_2016}). For images that have a variable ratio of instrumental to confusion noise like the \textit{Herschel} images of H-ATLAS, \citealt{Chapin_2011} showed that a convolving function or "matched filter" can be calculated to provide the maximum SNR for an unresolved source.

To detect H-ATLAS sources from the 250\,\micron maps using a matched filter (the 250\,\micron band is the most sensitive of the SPIRE bands and given the lower sensitivity of the PACS instrument, all sources detected on the PACS images would also be detected on the SPIRE 250\,\micron image), \citealt{Maddox_2020} developed a source detection algorithm called the Multi-band Algorithm for Source Detection and eXtraction (MADX). The MADX algorithm works in the following way. Firstly, Galactic dust emission is removed from the images using \texttt{Nebuliser}. Next, the images are convolved with the matched filter {\color{orange}(which needs description)}. The variance map is created by convolving the map of variance in instrumental noise with the matched filter and adding the confusion noise. It is from this map that the SNR of a detected source is determined. The same process is repeated with the 350 and 500\,\micron maps and interpolated to the same pixel scale as the 250\,\micron maps. The detection map used to extract sources is then generated from a weighted sum of the three SPIRE maps, however, due to the smaller PSF at 250\,\micron which leads to more accurate positions and the increased number of sources when using the 250\,\micron maps, zero weighting is given to the 350 and 500\,\micron images. This has the effect of making the detection map the same as the 250\micron map.

Sources are identified by peak values > 2.5$\sigma$ in the filtered detection map. Their positions are estimated by fitting a Gaussian to the nearest pixels surrounding the location of the peak. The source is extracted in the other \textit{Herschel} wavebands at the 250\,\micron position. Due to the high levels of confusion and high source density on the SPIRE maps, the flux density estimates in each band can be biased by blending with other sources. The MADX algorithm negates some of this problem by ordering the sources by their flux density estimates and iteratively fitting and removing a point source from the position of each source, starting with the brightest. The new estimates of the flux densities are then not influenced by contamination from brighter sources.

The catalogue of point sources provided by H-ATLAS come from the extraction of point sources using MADX applied to the SPIRE images of the NGP, SGP and GAMA fields. The final sources list is reduced to those sources with SNR > 4 in any of the SPIRE bands. While the detection method suggests that we may miss sources that are faint at 250\,\micron but bright at 350 or 500\,\micron, due to the weighting of the three images, cataloguing all sources with SNR > 4 in any of the SPIRE bands means that the catalogues are reasonably complete in all bands. The completeness of the sub-mm catalogues as a function of the measured flux density of a source as estimated by \citealt{Valiante_2016} is illustrated in Figure \ref{fig:submm_completeness}.

\begin{figure}
	\includegraphics[width=\columnwidth]{Figures/submm_completeness.pdf}
	\caption{Caption}
	\label{fig:submm_completeness}
\end{figure}

\subsection{Data Releases of the H-ATLAS}
\label{sec:Data Releases of the H-ATLAS}

The first public data release (DR1) of H-ATLAS covered the three equatorial GAMA fields, which span approximately 25\,\% of the total survey area. These fields benefit from multiwavelength coverage from GAMA, SDSS, 2dF, the Galaxy Evolution Explorer (GALEX, \citealt{Martin_2005}), the UKIRT Infrared Deep Sky Survey -- Large Area Survey (UKIDSS-LAS, \citealt{Lawrence_2007}), the Wide-field Infrared Survery Explorer (WISE, \citealt{Wright_2010}), the VISTA Kilo-degree Infrared Galaxy survey (VIKING, \citealt{Edge_2013}) and the Kilo-Degree Survey (KiDS, \citealt{deJong_2013}). 

Sources are provided with DR1 if they are detected above the 2.5$\sigma$ detection limit on the 250\,\micron map and have measured flux densities greater than the 4$\sigma$ flux density limits in one of the three SPIRE bands (29.6\,mJy, 37.6\,mJy or 40.8mJy at 250, 350 and 500\,\micron). Across the three fields there are a total of 113,995, 46,209 and 11,011 sources detected at > 4$\sigma$ at 250, 350 and 500\,\micron as well as detections for 4,650 and 5,685 sources at > 3$\sigma$ at 100 and 160\,\micron (\citealt{Valiante_2016}). Following the release of the sub-mm sources detected in the GAMA fields, \citealt{Bourne_2016} used the Likelihood Ratio (LR) method (Section \ref{sec:The Likelihood Ratio Method}) to identify potential optical counterparts to the 113,995 sources with SNR$_{250}$ > 4 from SDSS. Sources with SNR$_{250}$ < 4 that were detected by their 350 or 500\,\micron flux densities were omitted from the matching since these sources have sub-mm colours suggesting a high redshift, and are the most likely sources to be misidentified by SDSS due to the increased probability of chance alignments or gravitational lensing along the line of sight (\citealt{Negrello_2010}; \citealt{Pearson_2013}; \citealt{Bourne_2014}). {\color{orange} Description of Bourne+2016 main findings}.

The second public data release (DR2) covered the NGP and SGP, two large fields that together form $\sim$\,75\,\% of the total survey area. The NGP was covered in the optical by the SDSS and in the near-infrared by UKIDSS-LAS. Moreover, a small area of 25.93\,$\deg^2$ within the NGP was also observed by a deeper K-band survey by the H-ATLAS team using UKIRT (limiting magnitude of K < 19.40 compared to K < 18.69 for UKIDSS-LAS). The SGP is the largest field (approximately half the survey area of H-ATLAS) and was covered by the 2dF spectroscopic survey, KiDS in four optical bands ($u$, $g$, $r$ and $i$) and VIKING in five near-infrared bands ($Z$, $Y$, $J$, $H$ and $K_s$).

Given that sub-mm sources are only extracted from areas of the \textit{Herschel} maps that have at least two obsersations from the SPIRE instrument, the DR2 catalogues includes sources from the map area reduced by the masking of single \textit{Herschel} scans. The mask reduces the area covered by the NGP point source catalogue to 177.1\,$\deg^2$ and the SGP to 303.4\,$\deg^2$. As with DR1, sources are included if they are detected on the 250\,\micron map above the 2.5$\sigma$ detection limit by the MADX algorithm and surpass at least one of the 4$\sigma$ flux density limits at the SPIRE wavelengths. The catalogues contain 118,980 sources for the NGP field (112,069, 48,876 and 10,368 detected at > 4$\sigma$ at 250, 350 and 500\,\micron and 5,036 and 7,046 at > 3$\sigma$ at 100 and 160\,\micron respectively) and 193,527 sources for the SGP field (182,282, 74,096 and 16,084 at 250, 350 and 500\,\micron and 8,598 and 11,894 at 100 and 160\,\micron). {\color{orange} Description of Furlanetto+2018 main findings and the 2MASS preliminary matching}.

\section{Identifying Multiwavelength Counterparts to Herschel Sources}
\label{sec:Identifying Multiwavelength Counterparts to Herschel Sources}

\subsection{The Likelihood Ratio Method}
\label{sec:The Likelihood Ratio Method}