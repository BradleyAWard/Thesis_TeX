\section{The Herschel-ATLAS}
The \textit{Herschel} Astrophysical Terahertz Large Area Survey (H-ATLAS; \citealt{Eales_2010}) was the largest open-time sub-mm survey carried out with \textit{Herschel}. The survey was observed across five photometric bands using two instruments on board \textit{Herschel}: the Photodetector Array Camera (PACS, \citealt{Poglitsch_2010}) at 100 and 160\,\micron, and the Spectral and Photometric Imaging Receiver (SPIRE, \citealt{Griffin_2010}) at 250, 350 and 500\,\micron.

{\color{orange} The SPIRE bands were able to detect the cold dust missed by previous observatories, while at a better angular resolution. Notably, compared to ground based telescopes at wavelengths >850\,\micron where low-z galaxies are intrinsically faint, the SPIRE bands span the SED peak of typical galaxies in the local Universe.}

{\color{orange}The main scientific goal of the H-ATLAS was to provide measurements of the dust masses and dust obscured star formation for tens of thousands of nearby galaxies (to create and FIR/sub-mm analogue to the SDSS). The original goal was to provide a shallow survey over a large area of sky, but the exceptional sensitivity of \textit{Herschel} and the negative k-correction at sub-mm wavelengths (References) means a significant fraction of sources lie at high redshifts (References).}

The complete survey covers $\sim$\,660\,$\deg^2$, split into three regions located to avoid emission from Galactic dust and to utilize complimentary spectroscopic surveys including the Sloan Digital Sky Survey (SDSS, \citealt{York_2000}), the 2df Galaxy Redshift Survey (2dfGRS, \citealt{Colless_2001}) and the Galaxy and Mass Assembly (GAMA, \citealt{Driver_2009}). The North Galactic Pole (NGP) region covers $\sim$\,180\,$\deg^2$ of the northern sky,