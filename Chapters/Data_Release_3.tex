\section{The Herschel-ATLAS}
The \textit{Herschel} Astrophysical Terahertz Large Area Survey (H-ATLAS; \citealt{Eales_2010}) was the largest open-time sub-mm survey carried out with \textit{Herschel}. The survey was observed across five photometric bands using two instruments onboard the \textit{Herschel Space Observatory}: the Photodetector Array Camera (PACS, \citealt{Poglitsch_2010}) at 100 and 160\,\micron, and the Spectral and Photometric Imaging Receiver (SPIRE, \citealt{Griffin_2010}) at 250, 350 and 500\,\micron. Compared to the first SMGs {\color{orange} (Assumption: I have already described SMGs and their initial discovery in the 80s/90s.)} detected using SCUBA at 850\,\micron (\citealt{Smail_1997}; \citealt{Barger_1998}; \citealt{Hughes_1998}), the PACS and SPIRE wavebands span the peak of the infrared spectrum for low redshift (z < 1) galaxies. Their intrinsic brightness at the SPIRE wavelengths makes their detection in the thousands more achievable. 

%{\color{orange}The main scientific goal of the H-ATLAS was to provide measurements of the dust masses and dust obscured star formation for tens of thousands of nearby galaxies (to create and FIR/sub-mm analogue to the SDSS). The original goal was to provide a shallow survey over a large area of sky, but the exceptional sensitivity of \textit{Herschel} and the negative k-correction at sub-mm wavelengths (References) means a significant fraction of sources lie at high redshifts (References). While the original goal was to study dust and newly formed stars hidden by dust in nearby (z < 0.4) galaxies, the negative k-correction mean that the median redshift of sources is z $\sim$ 1. The source catalogues include sources up to a redshift of at least 6 (References).}

The complete survey covers $\sim$\,660\,$\deg^2$, split into three regions located to avoid emission from Galactic dust and to utilize complimentary spectroscopic surveys including the Sloan Digital Sky Survey (SDSS, \citealt{York_2000}), the 2df Galaxy Redshift Survey (2dfGRS, \citealt{Colless_2001}) and the Galaxy and Mass Assembly (GAMA, \citealt{Driver_2009}). The North Galactic Pole (NGP) region covers $\sim$\,180\,$\deg^2$ of the northern sky, centered at R.A 13$^{h}$18$^{m}$ and declination +29$^{\circ}$13' (J2000); three equatorial fields, located at approximately R.A 9$^{h}$, 12$^{h}$ and 15$^{h}$ coinciding with the GAMA survey (henceforth named GAMA9, GAMA12 and GAMA15 fields), each with an area of approximately 54\,$\deg^2$, and the South Galactic Pole (SGP) region, centered at R.A 0$^{h}$6$^{m}$ and declination -32$^{\circ}$44' (J2000) with an area of $\sim$ 318\,$\deg^2$. 

\subsection{Detecting Submillimeter Sources on Herschel Images}

Due to {\color{orange}[...]} sub-mm images suffer from two types of noise; instrumental noise {\color{orange}[...]} and confusion noise which is highly correlated between pixels, most of its contribution coming from the blending together of faint sources. Source confusion is of particular importance to sub-mm surveys {\color{orange}[...]}. The result of combining instrumental noise with confusion noise is that almost all sources in the Herschel images are unresolved and the optimum filter for detecting these unresolved sources is no longer the point spread function (PSF). Consider a \textit{Herschel} map in which there is only one source of noise: an image with instrumental noise but no confusion noise (i.e. there is only one point source and no fainter, confusing sources), the optimal detection of this source is obtained by convolving the image with the PSF of the instrument. On the other hand, a map with no instrumental noise, but many confused point sources would be optimally detected with its best signal to noise ratio (SNR) by taking the Fourier transform of the image, dividing by the Fourier transform of the PSF and taking the inverse Fourier transform to obtain a perfect deconvolution of the original map {\color{orange}(Reference)}. For images that have a variable ratio of instrumental to confusion noise like the \textit{Herschel} images of H-ATLAS, {\color{orange}Chapin et al, 2011} showed that a convolving function or "matched filter" can be calculated to provide the maximum SNR for an unresolved source.

To detect H-ATLAS sources from the 250\,\micron maps using a matched filter (the 250\,\micron band is the most sensitive of the SPIRE bands and given the lower sensitivity of the PACS instrument, all sources detected on the PACS images would also be detected on the SPIRE 250\,\micron image), {\color{orange} Maddox et al, 2020} developed a source detection algorithm called the Multi-band Algorithm for Source Detection and eXtraction (MADX).

\subsection{Data Releases of the H-ATLAS}

\section{Identifying Multiwavelength Counterparts to Herschel Sources}

\subsection{The Likelihood Ratio Method}