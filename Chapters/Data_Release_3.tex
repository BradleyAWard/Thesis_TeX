\section{The Herschel-ATLAS}
\label{sec:The Herschel-ATLAS}

The \textit{Herschel} Astrophysical Terahertz Large Area Survey (H-ATLAS; \citealt{Eales_2010}) was the largest open-time sub-mm survey carried out with \textit{Herschel}. The survey was observed across five photometric bands using two instruments onboard the \textit{Herschel Space Observatory}: the Photodetector Array Camera (PACS, \citealt{Poglitsch_2010}) at 100 and 160\,\micron, and the Spectral and Photometric Imaging Receiver (SPIRE, \citealt{Griffin_2010}) at 250, 350 and 500\,\micron. Compared to the first SMGs detected using SCUBA at 850\,\micron (\citealt{Smail_1997}; \citealt{Barger_1998}; \citealt{Hughes_1998}), the PACS and SPIRE wavebands span the peak of the infrared spectrum for low redshift (z < 1) galaxies. Their intrinsic brightness at the SPIRE wavelengths makes their detection in the thousands more achievable. The main scientific goal of the survey was to estimate the dust masses and dust obscured star formation rates for thousands of nearby galaxies over a large area of sky. While the intention was for a shallow survey, the surprising sensitivity of \textit{Herschel} and the negative k-correction observed at the operating wavelengths of the SPIRE instrument (\citealt{Blain_1993}) means that many sources were observed at higher redshifts, with a median of z $\sim$ 1. The catalogues of the survey, as detailed below, includes sources with redshifts up to $\sim$ 6 (\citealt{Amblard_2010}; \citealt{Lapi_2011}; \citealt{Fudamoto_2017}; \citealt{Zavala_2018}).

The complete survey covers $\sim$\,660\,$\deg^2$, split into three regions located to avoid emission from Galactic dust and to utilize complimentary spectroscopic surveys including the Sloan Digital Sky Survey (SDSS, \citealt{York_2000}), the 2df Galaxy Redshift Survey (2dfGRS, \citealt{Colless_2001}) and the Galaxy and Mass Assembly (GAMA, \citealt{Driver_2009}). The North Galactic Pole (NGP) region covers $\sim$\,180\,$\deg^2$ of the northern sky, centered at R.A 13$^{h}$18$^{m}$ and declination +29$^{\circ}$13' (J2000); three equatorial fields, located at approximately R.A 9$^{h}$, 12$^{h}$ and 15$^{h}$ coinciding with the GAMA survey (henceforth named GAMA9, GAMA12 and GAMA15 fields), each with an area of approximately 54\,$\deg^2$, and the South Galactic Pole (SGP) region, centered at R.A 0$^{h}$6$^{m}$ and declination -32$^{\circ}$44' (J2000) with an area of $\sim$ 318\,$\deg^2$. 

\subsection{Detecting Submillimeter Sources on Herschel Images}
\label{sec:Detecting Submillimeter Sources on Herschel Images}

Due to [...]\todo[color=orange]{continuation:} sub-mm images suffer from two types of noise; instrumental noise [...]\todo[color=orange]{continuation:} and confusion noise which is highly correlated between pixels, most of its contribution coming from the blending together of faint sources. Source confusion is of particular importance to sub-mm surveys [...]\todo[color=orange]{continuation:}. The result of combining instrumental noise with confusion noise is that almost all sources in the Herschel images are unresolved and the optimum filter for detecting these unresolved sources is no longer the point spread function (PSF). Consider a \textit{Herschel} map in which there is only one source of noise: an image with instrumental noise but no confusion noise (i.e. there is only one point source and no fainter, confusing sources), the optimal detection of this source is obtained by convolving the image with the PSF of the instrument. On the other hand, a map with no instrumental noise, but many confused point sources would be optimally detected with its best signal to noise ratio (SNR) by taking the Fourier transform of the image, dividing by the Fourier transform of the PSF and taking the inverse Fourier transform to obtain a perfect deconvolution of the original map (\citealt{Valiante_2016}). For images that have a variable ratio of instrumental to confusion noise like the \textit{Herschel} images of H-ATLAS, \citealt{Chapin_2011} showed that a convolving function or "matched filter" can be calculated to provide the maximum SNR for an unresolved source.

To detect H-ATLAS sources from the 250\,\micron maps using a matched filter (the 250\,\micron band is the most sensitive of the SPIRE bands and given the lower sensitivity of the PACS instrument, all sources detected on the PACS images would also be detected on the SPIRE 250\,\micron image), \citealt{Maddox_2020} developed a source detection algorithm called the Multi-band Algorithm for Source Detection and eXtraction (MADX). The MADX algorithm works in the following way. Firstly, Galactic dust emission is removed from the images using \texttt{Nebuliser}. Next, the images are convolved with the matched filter [...]\todo[color=yellow]{continuation: description of matched filters}. The variance map is created by convolving the map of variance in instrumental noise with the matched filter and adding the confusion noise. It is from this map that the SNR of a detected source is determined. The same process is repeated with the 350 and 500\,\micron maps and interpolated to the same pixel scale as the 250\,\micron maps. The detection map used to extract sources is then generated from a weighted sum of the three SPIRE maps, however, due to the smaller PSF at 250\,\micron which leads to more accurate positions and the increased number of sources when using the 250\,\micron maps, zero weighting is given to the 350 and 500\,\micron images. This has the effect of making the detection map the same as the 250\micron map.

Sources are identified by peak values > 2.5$\sigma$ in the filtered detection map. Their positions are estimated by fitting a Gaussian to the nearest pixels surrounding the location of the peak. The source is extracted in the other \textit{Herschel} wavebands at the 250\,\micron position. Due to the high levels of confusion and high source density on the SPIRE maps, the flux density estimates in each band can be biased by blending with other sources. The MADX algorithm negates some of this problem by ordering the sources by their flux density estimates and iteratively fitting and removing a point source from the position of each source, starting with the brightest. The new estimates of the flux densities are then not influenced by contamination from brighter sources.

The catalogue of point sources provided by H-ATLAS come from the extraction of point sources using MADX applied to the SPIRE images of the NGP, SGP and GAMA fields. The final sources list is reduced to those sources with SNR > 4 in any of the SPIRE bands. While the detection method suggests that we may miss sources that are faint at 250\,\micron but bright at 350 or 500\,\micron, due to the weighting of the three images, cataloguing all sources with SNR > 4 in any of the SPIRE bands means that the catalogues are reasonably complete in all bands. The completeness of the sub-mm catalogues as a function of the measured flux density of a source as estimated by \citealt{Valiante_2016} is illustrated in Figure \ref{fig:submm_completeness}.

\begin{figure}
	\includegraphics[width=\columnwidth]{Figures/submm_completeness.pdf}
	\caption{The completeness of the H-ATLAS Data Release I catalogues of sub-mm sources, as a function of the measured flux density at 250\,\micron (black) 350\,\micron (red) and 500\,\micron (blue). This figure is replotted from Figure 21 in \citealt{Valiante_2016}.}
	\label{fig:submm_completeness}
\end{figure}

\subsection{Data Releases of the H-ATLAS}
\label{sec:Data Releases of the H-ATLAS}

The first public data release (DR1) of H-ATLAS covered the three equatorial GAMA fields, which span approximately 25\% of the total survey area. These fields benefit from multiwavelength coverage from GAMA, SDSS, 2dF, the Galaxy Evolution Explorer (GALEX, \citealt{Martin_2005}), the UKIRT Infrared Deep Sky Survey -- Large Area Survey (UKIDSS-LAS, \citealt{Lawrence_2007}), the Wide-field Infrared Survery Explorer (WISE, \citealt{Wright_2010}), the VISTA Kilo-degree Infrared Galaxy survey (VIKING, \citealt{Edge_2013}) and the Kilo-Degree Survey (KiDS, \citealt{deJong_2013}). 

Sources are provided with DR1 if they are detected above the 2.5$\sigma$ detection limit on the 250\,\micron map and have measured flux densities greater than the 4$\sigma$ flux density limits in one of the three SPIRE bands (29.6\,mJy, 37.6\,mJy or 40.8mJy at 250, 350 and 500\,\micron). Across the three fields there are a total of 113,995, 46,209 and 11,011 sources detected at > 4$\sigma$ at 250, 350 and 500\,\micron as well as detections for 4,650 and 5,685 sources at > 3$\sigma$ at 100 and 160\,\micron (\citealt{Valiante_2016}). Following the release of the sub-mm sources detected in the GAMA fields, \citealt{Bourne_2016} used the Likelihood Ratio (LR, \citealt{Sutherland_1992}; \citealt{Ciliegi_2003}) method (Section \ref{sec:The Likelihood Ratio Method}) to identify potential optical counterparts to the 113,995 sources with SNR$_{250}$ > 4 from SDSS. Sources with SNR$_{250}$ < 4 that were detected by their 350 or 500\,\micron flux densities were omitted from the matching since these sources have sub-mm colours suggesting a high redshift, and are the most likely sources to be misidentified by SDSS due to the increased probability of chance alignments or gravitational lensing along the line of sight (\citealt{Negrello_2010}; \citealt{Pearson_2013}; \citealt{Bourne_2014}). \citealt{Bourne_2016} found optical counterparts within 10" of 44,385 (39\%) sources with an estimated probability of being the true ID > 80\% (the probability of an optical or near-infrared object being the true counterpart to a sub-mm source is defined as the reliability, R, and is derived in Section \ref{sec:The Likelihood Ratio Method}).

The second public data release (DR2) covered the NGP and SGP, two large fields that together form $\sim$\,75\% of the total survey area. The NGP was covered in the optical by the SDSS and in the near-infrared by UKIDSS-LAS. Moreover, a small area of 25.93\,$\deg^2$ within the NGP was also observed by a deeper K-band survey by the H-ATLAS team using UKIRT (limiting magnitude of K < 19.40 compared to K < 18.69 for UKIDSS-LAS). The SGP is the largest field (approximately half the survey area of H-ATLAS) and was covered by the 2dF spectroscopic survey, KiDS in four optical bands ($u$, $g$, $r$ and $i$) and VIKING in five near-infrared bands ($Z$, $Y$, $J$, $H$ and $K_s$).

Given that sub-mm sources are only extracted from areas of the \textit{Herschel} maps that have at least two obsersations from the SPIRE instrument, the DR2 catalogues includes sources from the map area reduced by the masking of single \textit{Herschel} scans. The mask reduces the area covered by the NGP point source catalogue to 177.1\,$\deg^2$ and the SGP to 303.4\,$\deg^2$. As with DR1, sources are included if they are detected on the 250\,\micron map above the 2.5$\sigma$ detection limit by the MADX algorithm and surpass at least one of the 4$\sigma$ flux density limits at the SPIRE wavelengths. The catalogues contain 118,980 sources for the NGP field (112,069, 48,876 and 10,368 detected at > 4$\sigma$ at 250, 350 and 500\,\micron and 5,036 and 7,046 at > 3$\sigma$ at 100 and 160\,\micron respectively) and 193,527 sources for the SGP field (182,282, 74,096 and 16,084 at 250, 350 and 500\,\micron and 8,598 and 11,894 at 100 and 160\,\micron). \citealt{Furlanetto_2018} applied the Likelihood Ratio method to all counterparts within 10" of the 250\,\micron sources of the NGP using both the shallower optical and near-infrared catalogue of SDSS and UKIDSS-LAS, and the deeper K-band survey. Of the 112,155 SPIRE sources with SNR$_{250}$ > 4, 77,521 (69.1\%) had at least one shallow optical counterpart and 42,429 (37.8\%) of these were matched with R > 0.8. In the smaller area observed with WFCAM, \citealt{Furlanetto_2018} identified 32,041 possible deep near-IR counterparts to 17,247 sources. 10,668 (61.9\%) of these sources were matched with an equally high reliability. While this analysis suggests that the inclusion of deeper K-band data drastically increases the fraction of sources matched to their corresponding optical or near-IR counterpart, [...]\todo[color=yellow]{continuation: we must consider their differences and think about Q}.

In the SGP a preliminary counterpart analysis was conducted using the Two Micron All Sky Survey (2MASS, \citealt{Skrutskie_2006}), but no formal LR analysis had yet been applied. A nearest neighbour match within 5" of a 2MASS galaxy gives identifications for 3,444 \textit{Herschel} sources. In the following section we detail the Likelihood Ratio method and apply it to the 250\,\micron sources detected by \textit{Herschel} in the SGP.

\subsection{Identifying Optical and Near-IR Counterparts to Herschel Sources}
\label{sec:Identifying Optical and Near-IR Counterparts to Herschel Sources}

When identifying multiwavelength counterparts across surveys the simplest choice to use the nearest neighbour within a fixed search radius of one of the sources. For surveys conducted at similar wavelengths with a similar resolution and sensitivity this is a suitable approach. However, when matching far-IR/sub-mm surveys to optical/IR data, the poor angular resolution of long wavelength instruments such as SPIRE (the FWHM of 250\,\micron detections with SPIRE is $\sim$ 18"), which cause large positional uncertainties, force us to increase the search radius around the sub-mm source. This effect, coupled with the intrinsic faintness of optical/near-IR counterparts due to dust obscuration, the relatively flat redshift distribution of sub-mm sources due to the k-correction and the high surface density of objects in optical/IR surveys, means that [...]\todo[color=yellow]{continuation: identification is not easy, Casey reference?} and it is common for there to be multiple possible counterparts within the search radius from a single sub-mm source.

Previously for sub-mm surveys it would be more practical to first match sources with radio or mid-IR sources and then use pre-existing matched catalogues to obtain multiwavelength data (e.g. \citealt{Ivison_2007}; \citealt{Dye_2009}; \citealt{Biggs_2011}, see also Section [...]\todo[color=green]{add reference for the final chapter}). However, presently this is not suitable for large surveys such as H-ATLAS as current radio telescopes do not provide the area and depth required to match with more than a small fraction of sub-mm sources. While current and future radio surveys from facilities such as the Square Kilometre Array (SKA), the Low Frequency Array (LOFAR) and MeerKAT will increase the radio coverage of the H-ATLAS fields, currently a statitstical identification method is still the preferred way of deciding which objects are associated and which are unrelated foreground/background objects to large samples of sub-mm sources.

\subsection{The Likelihood Ratio Method}
\label{sec:The Likelihood Ratio Method}

The Likelihood Ratio method assigns a probability (reliability) to all potential matches surrounding low resolution sources to distinguish between likely counterparts and chance alignments and has been used many times to identify counterparts to \textit{Herschel} sources. The LR method was used by \citealt{Smith_2011} to identify SDSS counterparts in the Science Demonstration Phase (SDP) catalogue (a preliminary data release for H-ATLAS, overlapping with the GAMA9 field), by \citealt{Kim_2012} to identify Spitzer-IRAC counterparts also in the SDP data, by \citealt{Fleuren_2012} for VIKING IDs in the Phase 1 catalogue of the GAMA9 field, and as mentioned earlier, by \citealt{Bourne_2016} and \citealt{Furlanetto_2018} to find optical and near-IR counterparts in the GAMA fields and NGP field respectively.

The likelihood, $L$, of a counterpart being the true identification to a \textit{Herschel} source is given by the ratio between the probability that an object observed at a given radius from the source, $r$, with an optical or near-IR magnitude, $m$, is the true identifcation and the probability of observing an unassociated object with the same $r$ and $m$. On the assumption that the distance from the source and the optical/near-IR magnitude are independent on their influence on the probability of being a true counterpart, we find that:

\begin{equation}
\label{eq:likelihood_ratio}
    L = \frac{P(\textrm{ID}, r, m)}{P(\textrm{unassociated}, r, m)} = \frac{P(\textrm{ID}, r) P(\textrm{ID}, m)}{P(\textrm{unassociated}, r, m)}
\end{equation}

Each term in the above equation can be defined in the following way: $f(r) \coloneqq P(\textrm{ID}, r)$, $q(m) \coloneqq P(\textrm{ID}, m)$ and $n(m) \coloneqq P(\textrm{unassociated}, r, m)$, where $f(r)$ represent the radial probability distribution function of positional errors between the source and counterpart, $q(m)$ represents the magnitude probability distribution of true counterparts and $n(m)$ is the magnitude distribution of background objects from the input survey. By using Baye's theorem and the theorem of total probability, we can define the probability that a counterpart is the true ID given it has $r$ and $m$ as:

\begin{equation}
\label{eq:reliability_one_counterpart}
    R \coloneqq P(\textrm{ID}| r, m) = \frac{L}{L+1}.
\end{equation}
\todo[color=green]{Requires a derivation}

Equation \ref{eq:reliability_one_counterpart} assumes that there is only a single candidate with a likelihood $L$. For a source with multiple possible candidates, the reliability $R_j$ of the $j^{th}$ candidate is given by:

\begin{equation}
    \label{eq:reliability_multiple_counterparts}
        R_j = \frac{L_j}{\sum_i L_i + (1-Q)},
\end{equation}

where $i$ represents the $i^{th}$ counterpart found within the search radius. The $Q$ parameter represents the fraction of all true counterparts that are brighter than the limiting magnitude of the input survey and can therefore be observed. This means that the (1 - $Q$) term represents the probability that the counterpart is not observed and accounts for the fact that not all counterparts will be detected in the optical/near-IR survey. The value of $Q$ depends on the depth of the survey and the choice of passband used. In the following sections I shall outline the methods used to estimate the functions $f(r)$, $q(m)$ and $n(m)$ and to estimate $Q$ to calculate the likelihood ratios and reliabilities of near-IR counterparts observed on the VIKING images surrounding the 250\,\micron positions of \textit{Herschel} sources in the SGP.

\section{Applying the LR Method to VIKING Galaxies in the SGP}
\subsection{VISTA VIKING Counterparts}

The Visible and Infrared Survey Telescope for Astronomy (VISTA) is a 4\,m wide field telescope located at the ESO Paranal Observatory in Chile. The telescope has five near-IR broad band filters, $Z$, $Y$, $J$, $H$ and $K_s$, that have central wavelengths between 0.88 and 2.15\,\micron (\citealt{Emerson_2010}). The VIKING survey was a public survey with VISTA, covering approximately 1,500\,$\deg^{2}$ of sky, including an overlap of more than 360\,$\deg^{2}$ with the H-ATLAS survey in the GAMA and SGP fields, to a 5\,$\sigma$ depth of 23.1, 22.3, 22.1, 21.5 and 21.2 (AB) in the above five filters.

We take as our object catalogue all objects observed in the fourth data release of VIKING within 15" of the 250\,\micron position of each \textit{Herschel} source. The counterpart matching in the GAMA9 field by \citealt{Fleuren_2012}, recovered 51\% of all 250\,\micron sources with a reliable (R > 0.8) VIKING counterpart. Compared to the optical r-band of the SDSS as used in \citealt{Bourne_2016} and \citealt{Furlanetto_2018} which have typical returns of $\sim$ 35 -- 40\% due to the limiting magnitude of SDSS, we expect the SGP to have reliable identifications for approximately half of all SGP sources. The SGP fields contains 193,527 sources detected at greater than 4\,$\sigma$ significance, suggesting that we might expect to match $\sim$ 100,000 \textit{Herschel} sources with a near-IR counterpart with a high probability.

However, a significant number of sources in the VIKING survey are stars that would be erroneously matched to H-ATLAS. The sub-mm emission from stars is most likely from debris discs or dust in outflows. As there is large variation in the mass and temperature of debris discs for stars of a given spectral type (\citealt{Hillenbrand_2008}), and \textit{Herschel} is only sensitive to the brightest of these discs (\citealt{Thompson_2010}), there is much scatter in the sub-mm properties of \textit{Herschel} detected stars which would result in poor statistics of dusty stars when calculating the likelihood of counterparts. For this reason, I use an adapted method of \citealt{Baldry_2010} to separate stars and galaxies in the VIKING SGP catalogue and apply the LR method separately for the two classes.

The method of \citealt{Baldry_2010} uses near-IR $J$ and $K_s$ and optical $g$ and $i$ bands to define a line of separation between stars and galaxies in $J - K_s$, $g-i$ colour-colour space, and is used in \citealt{Bourne_2016} and \citealt{Furlanetto_2018} to separate stellar and extragalactic objects in SDSS. Without coverage from SDSS in the SGP, I use the fourth data release of KiDS to identify optical $g$ and $i$ bands for [...]\todo[color=green]{Look up value} of our VIKING sources. A nearest neighbour search to a maximimum of 0.5" from the 250\,\micron position of each source was used.

First, I classify as stellar any object in our catalogue with \texttt{pStar} > 0.95, an estimate of the probability that the source is a star, based on a shape parameter provided as part of the VIKING data release. This immediately classifies 51,508 objects as stars. Next I consider the $J - K_s$, $g-i$ colour-colour space and define a stellar locus by converting the locus in \citealt{Baldry_2010} to the Vega system assuming $J_{\textrm{Vega}}$ = $J_{\textrm{AB}}$ - 0.91 and $K_{s,\textrm{Vega}}$ = $K_{s,\textrm{AB}}$ - 1.85:

\begin{equation}
    f_{\textrm{locus}} = 
    \begin{cases*}
        0.228 & $g-i$ < 0.3 \\
        0.05 + 0.615(g-i) - 0.13(g-i)^2 & 0.3 $\leq$ $g-i$ < 2.3 \\
        0.7768 & $g-i$ $\geq$ 2.3.
    \end{cases*}
\label{eq:stellar_locus}
\end{equation}

I define a line of separation between stars and galaxies as being +0.2 offset in $J - K_s$ from the stellar locus. The distribution of sources, stellar locus and separation line are illustrated in Figure \ref{fig:star_galaxy_classification}. This classifies a further 411,463 sources that have KiDS identifications (299,525 extragalactic and 111,938 stellar). The remaining objects do not have matches in KiDS and thus do not have $g$ and $i$-band magnitudes. However, based on the separation line defined above, it can be seen that those objects with $J - K_s$ < 0.42 will always fall below the separation line regardless of their optical colour, and similarly, any object with  $J - K_s$ > 0.98 will always lie above the line. The cross contamination of stars above the line and galaxies below the line is small, so I next define all remaining sources using the above single flux cuts. This classifies a further 102,540 sources (102,265 as galaxies and 275 as stars). Finally, I return to the \texttt{pStar} parameter and relax the criteria to those objects with \texttt{pStar} > 0.7; all other objects are classified as galaxies. This method leads to the identifcation of 793,331 (78.9\%) extragalactic sources and 212,028 (21.1\%) stars.

\begin{figure}
	\includegraphics[width=\columnwidth]{Figures/star_galaxy_classification.pdf}
	\caption{The $J - K_s$, $g-i$ colour-colour diagram of VIKING objects with KiDS identifications in the SGP. The stellar locus defined by Equation \ref{eq:stellar_locus} is illustrated as the blue line, while the separation between stars and galaxies, defined as +0.2 offset from the stellar locus, is shown as a dashed blue line. Extragalactic and stellar sources identified from our classification are shown as grey and red points respectively.}
	\label{fig:star_galaxy_classification}
\end{figure}

\subsection{True Counterpart Distribution, q(m)}

To estimate the reliability of each VIKING source being the true ID to a \textit{Herschel} source, I determine the probability distribution of true counterparts as a function of $K_s$-band magnitude, $q(m)$, using the method described in \citealt{Ciliegi_2003}. First, a magnitude distribution of all objects found in the VIKING images within 15" of a \textit{Herschel} source is generated separately for stars and galaxies, which we shall denote as $n'_{\textrm{total}}(m)$. Here I have set prime notation to reference raw counts, while no prime notation is reserved for counts that have been normalized by the total area searched on the VIKING images. The excess of sources above the background level in regions surrounding the 250\,\micron sources is an estimate for the number density of true VIKING associations and is estimated by subtracting the magnitude distribution for the whole VIKING survey from $n'_{\textrm{total}}(m)$. A set of 844,715 random positions were located on the SGP map and used to search for objects on the VIKING images to within 15". A total of 2,917,214 objects were found and form our $n'_{\textrm{background}}(m)$ distribution. The excess of "real" counterparts thus has a magnitude distribution that may be expected to follow

\begin{equation}
    n_{\textrm{real}}(m) = n'_{\textrm{total}}(m) - n'_{\textrm{background}}(m) \frac{N_{\textrm{250\,\micron}}}{N_{\textrm{background}}},
\end{equation}

where the distribution has been normalized to the search area by $N_{\textrm{250\,\micron}}$ and $N_{\textrm{background}}$, the number of 250\,\micron positions and the number of randomly located positions respectively.

I then derive the $q(m)$ distribution by normalizing $n_{\textrm{real}}(m)$ and scaling by the fraction of all true counterparts that would be visible on the VIKING images, $Q$. This ensures that the integral of $q(m)$ over all magnitudes up to the limiting magnitude of the VIKING survey is equal to the probability that the sources is detected, i.e. $\int^{m_{\textrm{lim}}} q(m)dm = Q$. This normalization can be written as:

\begin{equation}
\label{eq:true_counterparts_distribution}
    q(m) = \frac{n_{\textrm{real}}(m)}{\sum_{m_i}n_{\textrm{real}}(m_i)}\times Q,
\end{equation}

where we have summed over the magnitude bins, $m_i$. The magnitude distribution of "true" counterparts to the 250\,\micron SGP sources is illustrated in the middle panel of Figure \ref{fig:true_counterparts_distribution} and shows the difference between the two distributions for extragalactic and stellar sources, requiring us to implement the LR method separately for the two classes. I also show in Figure \ref{fig:true_counterparts_distribution} the magnitude distributions of $n_{\textrm{total}}$ and $n_{\textrm{background}}$ and the distributions $q(m)/n(m)$ (here $n(m)$ references the area-normalized background distribution of sources, as used in Equation \ref{eq:likelihood_ratio} to calculate the likelihood values). The calculation of the likelihood value for each VIKING source thus depends on an estimate for the fraction of true IDs that are observed on the VIKING images, $Q$, which I detail in the following section.

\begin{figure}
    \centering
	\includegraphics[height=0.75\textheight]{Figures/true_counterparts_distribution.pdf}
	\caption{Top panel: The $K_s$-band magnitude distributions of objects located within 15" of 250\,\micron \textit{Herschel} positions (blue solid line) and random positions (blue dashed line), also separated by extragalactic (black lines) and stellar (red lines) identifications. Middle panel: The $K_s$-band magnitude distribution of "true" counterparts accounting for the excess of VIKING sources observed near \textit{Herschel} sources. Bottom panel: The ratio between the "true" counterparts distribution (middle panel) and the background distribution of sources, as used in the calculation of likelihoods; Equation \ref{eq:likelihood_ratio}. The dashed lines represent smoothed fits to $q(m)/n(m)$ to provide a continuous function at all magnitudes. The colour convention of the middle and bottom panels are the same as the top panel.}
	\label{fig:true_counterparts_distribution}
\end{figure}

\subsection{Estimating Q}
\subsection{The Positional Offset Distribution, f(r)}

\listoftodos